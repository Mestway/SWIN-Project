\begin{section}{From Core SWIN to Full SWIN% Extending to Full Java
  }
\label{sec:extension}

In this section, we present the way to extend core SWIN on
Featherweight Java to full SWIN on full Java language formally.
Generally, the extension is based on the term extension and type
extension. By extending source code pattern and target code pattern to
a term in full Java and extending types to full Java in variable
declaration part of update rules, we are able to match a Java term and
then transform it to a term with new API by meta-variable
substitution.
\par
Extending SWIN to full SWIN, we need some special treatments of the following key points :

\paragraph{Package} % In full Java, $\tt import$ command allows us to
                    % import a package at the head of a program and
                    % use classes in the package without using the
                    % full name. In this case, when we update APIs in
                    % the package, we will match the short name if it
                    % is not ambiguous and transform it to a short
                    % name in target API, and then generate a $\tt
                    % import$ command in the head of target client
                    % code.
Full Java supports the \code{package} and \code{import} commands for name organization. To ease the writing
of transformation rules, we also support \code{import} command in
SWIN, yet all internal processing is based on fully qualified names.

%<<<<<<< .mine
%\paragraph{Assignment} 
%When we extend SWIN to full Java with assignment, we need to consider the problem with side effect.
%To keep the update process still holding type safety, we need to distinguish the difference the application of $\tt\Pi$ on a L-value between that on a R-value.
%We need to check that an update rule will not transform a L-value to a R-value. And when we apply SWIN on an assignment expression, we simply apply it on both sides. 
%=======
\paragraph{Field and Assignment} 
%When we extend SWIN to full Java with assignment, we need to consider
%the problem with side effect.
FJ has no assignment statements and all fields are read-only. When assignments are introduced, expressions in Java can be
distinguished into L-value and R-value. To ensure type safety, we need
to ensure the transformation does not change an L-value into an
R-value. The most common L-value is field access. For example, given ``\code{a.x = b}'', if a transformation rule
transforms ``\code{a.x}'' into ``\code{new A()}'', the new code will fail to
compile because ``\code{new A()}'' is not a L-value. This check can be
implemented by applying the Java rules for distinguishing L-value and
R-value on the source patterns and the destination patterns.
%>>>>>>> .r506

% To keep the update process still holding safety property, we need to distinguish the difference the application of $\tt\Pi$ on a L-value between that on a R-value.
% We need to check that an update rule will not transform a L-value to a R-value. And when we apply SWIN on an assignment expression, we simply apply it on both sides. 

\paragraph{Static Method Access} In full Java, a method can be defined
as a static method, and we can access it by $\tt C.m(a,b,...)$. 
We treat the application of full SWIN on static method access as a normal method invoke, except that we 
need to apply the term directly on the class identifier. As the transformation of a class definition is 
by class name replacement, type safety can be guaranteed.

\paragraph{Overload} % Method overloading in full Java may lead to problems in the matching process.
% As we allow a meta-variable $\tt x$ of type $\tt A$ in source code
% pattern matches a term $\tt t$ with a subtype of $\tt A$, this may
% lead a piece of client code match more than one update rules. And this
% ambiguity can be resolved if we define a matching order on a FJ term
% with subtyping relation: we will always choose one with closest
% subtyping relation to match the term.  For
% example, if we have a relationship $\tt A<:B<:C$, and in class $\tt
% D$, we have methods $\tt f(B\ x)$ and $\tt f(C\ x)$. Then $\tt(new\ D()).f(new\
% A())$ will be matched to the first pattern as they have a closer
% subtyping relationship.
When method overloading is considered, we need to match a method not
only using its name, but also the type of its input parameters. Also,
the subtyping relation should be considered in the same way as Java:
when there are several overloaded methods that can be matched, we
choose the one with the closest subtyping relation on the parameters. For
example, if we have a relationship $\tt A<:B<:C$, and in class $\tt
D$, we have methods $\tt f(B\ x)$ and $\tt f(C\ x)$. Then $\tt(new\ D()).f(new\
A())$ is a call to the first method as they have a closer
subtyping relationship. A pattern matching $\tt f(C\ x)$ should not match
this term.

\paragraph{Generics} Generics in full Java affects the evaluation rules E-CLASS
and E-T-NEW. We have two extending rules to solve this problem.
\begin{enumerate}
    \item During pattern matching, a rule matches a generic type without considering
        parameterized types.
    \item After performing transformation on the generic type, 
        our rules recursively visit the parameterized types.
\end{enumerate}
% These two rules can assure the well-typedness of a term with parameterized types during transformation.
% The limitation of this way is that the generic class should not changes the number of
% type parameters during evolution which is reasonable in real APIs evolution.
The type safety is guaranteed because we require the preservation of
subtyping relation, and thus the constraints on generic parameters
will not be broken. The limitation of this method is that the number
of type parameters in a generic class cannot change from the source
API to the target API. However, we have never observe this kind of
change in practice.
\end{section}


%%% Local Variables: 
%%% mode: latex
%%% TeX-master: "pepm-15"
%%% End: 
