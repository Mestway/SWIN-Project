\section{Related work}
\label{sec:related}
% DSL or meta-programming

\smalltitle{General Transformation Frameworks} A number of
general-purpose program transformation languages/frameworks have been proposed.
To be independent of any programming languages, most of these languages
work on the grammatical level, defining transformations on top of
syntax trees. For example, TXL~\cite{txl} and
Stratego/XT~\cite{stratego} are general-purpose and grammar-oriented
transformation languages, which allow the definitions of a set of
rules to rewrite the abstract syntax trees of a program. Tom~\cite{tom} is a language extension for Java designed
to manipulate tree structures. In Tom, term rewriting and plain Java
code can be mixed to write more powerful program transformations.%  Tom can be used in Java to provide
% pattern matching like rule-based language. Any Java program is also a
% Tom program, and users can use both the extending features and Java
% language itself to perform transformation.
Compared with
these general-purpose transformation languages, SWIN mainly focuses on
transforming Java programs in the scope of API evolution and API
switching. By using Java features, SWIN allows more concise programs
to be written for these tasks. Furthermore, none of the general
transformation languages guarantees type-safety, for type-safety is
difficult to specify in a language-independent way.

\smalltitle{Transformation Frameworks for Java} Besides
Twinning~\cite{twinning}, 
several transformation languages/frameworks for Java
programs are proposed. For example, Spoon~\cite{fool06} 
is a transformation framework for Java programs, providing the ability
to directly read and modify program elements in Java
programs. As far as we know, these transformation frameworks for Java do
not consider type safety either, and there is no guarantee that the
transformation does not introduce compilation errors. Refaster~\cite{refaster}
uses compilable before-and-after examples of Java code to specify a Java
refactoring. Similarly as our work, this work also mainly focuses on solving the
method replacement which is useful in real API migration. Moreover, using 
direct Java examples to describe the transformation is convenient. However,
refaster cannot assure the well-typedness of the whole program during transformation, 
         as it only requires that each transformed expression is well typed.


\smalltitle{Type-Safe Transformations} Approaches for ensuring type
safety also exist. Hula~\cite{hula} is a rule-based update (or transformation) language for
Haskell, ensuring updates are performed in a type-safe manner. The
type-safe transformation depends on an core calculus--update
calculus~\cite{updatecalculus}, which provides type-safe
transformation over lambda calculus. This work distinguishes program
changes into declaration changes, definition changes, and application
changes, and requires the three changes to be consistent. Compared
with our work, update calculus allows the dynamic change of type
definitions during transformation while our approach focuses on static
type mappings as the difference between the old API and the new API
are already known during program adaptation for different APIs. On the
other hand, update calculus allows only the replacement of a type to a
more generic type, while our approach supports more type mapping
between independent types, such as Vector to ArrayList, because these
type changes are dominant in program adaptation between APIs.

The work of Balaban~\cite{oopsla05} et al. focuses on a particular
problem in the adaptation of program between APIs: when some part of
the program cannot be changed, how to change the other parts 
while preserving well-typedness and other properties. This work extracts the
type constraints from the Java program, then solves the constraints
using a constraint solver to prevent type incorrect program
transformations. Different from this work
that considers the well-typedness of a particular client program, our
work focuses on the type-safety of the transformation itself, taking
into account all possible client programs. The work of Spoon~\cite{fool06} 
focuses on the well-typedness of a program using API with forthcoming 
or deprecated methods. This work extends FJ with forthcoming
and deprecated methods, and proves the soundness of extended FJ.
However, this work only allows update on methods, rather than
update on classes.
% use the constraint slover to
% preserve the type correctness while updating programs. This work
% mainly focus on migrating programs used legacy API becuase some legacy
% API cannot be transformed.  This work extracted the type constraints
% from Java source code and rules, then solve the constraints using a
% constraints solver to prevent type incorrect program transformation
% performed with the rules.  Our work mainly focus on type incorrectness
% produced by API evolution and API alternative, and our work can
% preserve the type-safety of any clients whlie being updated by SWIN
% rules, rather than the type-safety of a specific client program as
% work showed by Balaban.

\smalltitle{Semantic-Preserving Transformations} Refactoring-based
approaches~\cite{reba, catchup} treat the API changes as a set of
refactorings. The API developers records their changes on the
API as a set of refactorings, and the later these refactorings can be
replayed on the client programs to transform the client programs to the new
API. In this way, the adaptation of the client programs is not only
type-safe but also semantic-preserving. However, this approach has
limitations. First, this approach cannot support API changes that
cannot be expressed as refactorings. Second, this approach only
applies to API update, and cannot support migrating programs between
alternative APIs, which are independently developed.
% operation-based approach\cite{reba, catchup} and
% matching-based approach\cite{dig-ecoop, graph-oopsla}. The operation-based approach is only suitable for
% API update. The approach is mainly use a tool to record the API update operation, and require
% the API update 
% operation only can be refactoring. 
% This approach can get the accurate API changes
% and it limits the API update operation only for
% refactoring. Our work allowed any API changes.  
The work of Leather~\cite{pepm14} et al. provides a approach 
to preserve semantics 
of a program while changing terms involving type A to terms involving
type B using type-changing rewrite rules. This work mainly
focuses on conversion between isomorphic types, whereas our work focuses on
transformation between any two types. Moreover, unlike this work performing
transformations on lambda calculus with let-polymorphism, our work performs
transformations on Featherweight Java which need to solve problems introducing
by object orientation, such as subtyping.

Package templates~\cite{packtemp} is an extension to Java to write
reusable and adaptable modules. Since the template instantiation
process in package templates includes operations like renaming and
class merging, it can be considered as a semantics-preserving program
transformation process. Different from our work, the program
transformation in package templates mainly focuses on the changes on the
class level, and does not consider the replacement of method
invocations. A key point in package templates is to avoid name
collision in transformation. Our approach does not consider this issue
because in Java language, the client code and the API are usually in
different packages, and the names are almost impossible to collide.

% solve a particular problem in
% semantic-preserving transformation: name collisions. When we merge
% classes and rename methods, methods with the same name need to be
% properly addressed 

% is a work focus on semantics-preserving source-to-source transformation of
% Java.
% While two classes merge or the name of methods changes, methods with same name
% may lead the behavior of client programs invoking these methods change. 
% This work mainly focus on the name problems while modifying programs, rather than on how
% to modify client programs type correctly while API evolution or API alternative as our
% work mainly foucs on. As there are lots of work focus on the problems caused by name changes,
% in our paper, we do not allow the name of variables change. \textit{We will give a brief check about 
% the form name of variables may introduce type errors in updated programs}


\smalltitle{Heuristic-based Transformations} Several approaches try to
further reduce the cost of program adaptation between APIs by
automatically discovering the transformation program using heuristic
rules. The heuristic rules range from comparing API source code~\cite{dig-ecoop},
analyzing existing client source
code~\cite{graph-oopsla,spatch,gpatch}, and discovering similar code
pieces~\cite{sediting}. Since these approaches are heuristic-based,
there is no guarantee the discovered transformations are type-safe.
  
% The work related to program adaptation in the 
% scope of API alternative or API updated can be split into
% three categories: using DSLs for program adaptation, acquire the API changes, 
% and apply the non-rule API changes to update client programs.

% There are lots of program transformation languages, 
% and most of these update languages are rule-based language. 
% TXL\cite{txl} and Stratego/XT\cite{stratego}
% are general-purpose and grammar-oriented transformation languages, 
% while apply rules on abstract syntax tree or concrete syntax tree of programs. 
% Comparing these general-purpose update languages,
% SWIN mainly focuses on Java program transformation in the scope of
% API evolution and API alternative. SWIN can express Java programs
% update more easily and conveniently for it use the Java features.
% Tom\cite{tom} 
% is a language extension designed to manipulate tree structures.
% Tom can be used in Java to provide pattern matching like rule-based language.
% Any Java program also are Tom program, 
% users can both use the extending features and Java language itself to perform
% transformation. 
% Although the expressiveness of these languages is more powerful, all these languages cannot
% preserving type-safety while transforming programs. SWIN is not so strong as these languages,
% but it useful in real experience, moreover, SWIN can preserving the type-safety while updating
% programs.

% The work of Balaban\cite{oopsla05} et al. use the constraint slover to preserve the type correctness 
% while updating programs. This work mainly focus on migrating programs used legacy API
% becuase some legacy API cannot be transformed.
% This work extracted the type constraints from Java source code and rules,
% then solve the constraints using a constraints solver to prevent type
% incorrect program transformation performed with the rules. 
% Our work mainly focus on type incorrectness produced by API evolution and API
% alternative, and our work can preserve the 
% type-safety of any clients whlie being updated by 
% SWIN rules, rather than the type-safety of a specific client program as work showed by Balaban.


% Hula\cite{hula} is a rule-based update language for Haskell, it perform the update in a type-safe manner. 
% The type-safe transformation mainly depend on an core calculus -- update calculus, and
% this work try to transform the hula to update calculus in a right way.
% Update calculus\cite{updatecalculus} is a general update language for lambda calculus, 
% and it can preserve the type correctness while updating programs.
% This work divided the changes into definition changes and application changes. 
% While methods' definitions change, the code applying these methods should be changed coordinately. 
% The type changes of method definition are dynamic, and need be inference. The type changes 
% of definition in our
% work are static, because we already known the updated API and alternative API.
% Our work mainly focus on how the application change can be right with defintion changes.
% However, the work of update calculus can only be allowed to transform a type
% to a more generic type, but our approaches
% allows more general type changes, such as Vector to ArrayList. 

% Package templates\cite{packtemp} 
% is a work focus on semantics-preserving source-to-source transformation of
% Java.
% While two classes merge or the name of methods changes, methods with same name
% may lead the behavior of client programs invoking these methods change. 
% This work mainly focus on the name problems while modifying programs, rather than on how
% to modify client programs type correctly while API evolution or API alternative as our
% work mainly foucs on. As there are lots of work focus on the problems caused by name changes,
% in our paper, we do not allow the name of variables change. \textit{We will give a brief check about 
% the form name of variables may introduce type errors in updated programs}


% Because our work is in the scope of API alternative and API update, work acquiring the
% API changes and apply API changes in clients are also related.
% Acquire the API changes can be also split in two categories: operation-based approach\cite{reba, catchup} and
% matching-based approach\cite{dig-ecoop, graph-oopsla}. The operation-based approach is only suitable for
% API update. The approach is mainly use a tool to record the API update operation, and require
% the API update 
% operation only can be refactoring. 
% This approach can get the accurate API changes
% and it limits the API update operation only for
% refactoring. Our work allowed any API changes.  
% The matching-based approach is more general, and this approach is 
% from a simple idea that if we want
% to get the changes of two things, we can just compare them. Using matching-based approach, first, we should
% process the program code, such as serve code as a text or AST (Abstract Syntax Tree). Then, we can use some
% approaches to compare the processed code.
% Dig\cite{dig-ecoop} treats the Java source
% code as text, they use the shingling algorithm from area of information retrieval to 
% encoding code text to a set of shingles. Then comparing two sets of shingles to get the differences. 
% The work of Nguyen et al.\cite{graph-oopsla} extract the AST from the Java APIs source code,
% then comparing the AST of these source code to get the differences. 
% This method is more accurate than just serve the source code as text for it use more code information.
% Although this work allowed any API updated, it cannot get the accurate API changes comparing 
% operation-based approach.


% Using the API changes to update the client programs with these API is a hard work. 
% The most frequently used approach is manually search the code need to update and replace the code
% to adapt to new API.
% The simplest search-and-replace is text match and regular expression match, but it is not enough for complex
% code modification, for these approaches cannot match the syntax-related code block.
% Semantic Grep\cite{sgrep}
% provide a approach that express the relations among packages, classes and methods. Using semantic grep, 
% users can express complex relations between packages, classes and methods, such as a class is belong to
% a package or a method called another method.
% It is much useful than common grep when user want to match the block with syntax relation.
% The work of Kapur et al. \cite{refactoring-replace}
% focus on the ability of replacements. The approaches above-mentioned all about the ability of match, 
% the replace operation is same for all the matched block. Therefore, Kapur
% provides a tool to refactoring replacements, such as adding a parameter for all the matched methods. 
% This work is useful that it extends replacements to refactoring, rather than simply replace.

% All above-mentioned about apply API changes to update programs need to be done manually.
% Manually update programs are tedious and error-prone. There are many work for automatically program adaptation.
% ReBA\cite{reba} and CatchUp!\cite{catchup} use the replay approaches to
% automatically update client programs. The premise of these approaches is to record the API changes when
% API vendor perform update, as mentioned before, the update operation can only be refactoring. After recording
% the update operation on APIs, clients developers can use these update operations to automatically update their
% program with the tool support.
% In order to apply the API changes to update program right.
% Nguyen et al. \cite{graph-oopsla} extract the data-flow graph and control-flow graph from programs, 
% and match the API changes from these
% two graph. These can help match the context-sensitive code block, and
% transform these block to new API in a right way. 
% Semantic patch\cite{spatch} and Generic patch\cite{gpatch} focus 
% on infer more general patches from existing patches of Linux kernel patches. 
% For the Linux kernel API evolution, the related Linux device drivers need to be update in response to 
% evolution.
% These approaches extracting the specific part of patches, and generic these parts. Then apply the more general
% patches to update device drivers automatically.
% Systematic editing\cite{sediting}
% focus on similar program update. Code clone is a common phenomenon in programs, so alleviate the work of similar
% code update is important. These work analyze the AST of programs, then get the update sequences of comparing two ASTs. 
% Finally, apply the update sequences to another similar programs.



%%% Local Variables: 
%%% mode: latex
%%% TeX-master: "pepm-15"
%%% End: 
