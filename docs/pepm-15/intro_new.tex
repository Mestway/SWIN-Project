\section{Introduction}
\label{intro}

% API switching is common
% API switching tool is often provided
% coding API switching tool is hard
%% in particular, type correctness is not guaranteed
% dedicated transformation tools are provided, but mainly syntactic
% achieving type safe is not easy
%% twinning does not achieve full type correctness
%% first, interleaving features are likely to be ignored
%% second, correctness and completeness need to be considered together

Modern programs often depend on different APIs (Application
Programming Interfaces), and it is a common task for the developers to
adapt programs between alternative APIs. One example is \emph{API update}: when an old API
is updated to a new version with incompatible changes, we need to
transform the client programs from the old API to the new API. Another
example is \emph{API switching}: we often need to migrate programs between different
platforms, such as from the Android platform to iOS, or from Java
Swing to SWT. In such cases, we need to transform the programs from the
API on one platform to the API on another platform. However, manually
adapting programs is not easy: we need to examine every use of the
source API and replace them with the suitable target API. Thorough
knowledge of the source and target APIs as well as the client program
is required.

Given the importance of program migration, it would be helpful and
beneficial for tool vendors to provide automated tools to assist
application adaptations. When API upgrades, API providers could
provide tools to automate the upgrade of client applications,
preventing potential loss of users from the incompatibility of the new
API. Similarly, platform providers could provide tools to facilitate
the migration of application from other platforms to their own
platforms, attracting more applications and users on their platform.
For example, Microsoft has provided the Visual Basic upgrade wizard
tool, to facilitate the transition from Visual Basic to Visual
Basic.Net. RIM has provided a tool suite to transform Android
applications into blackberry applications. These tools work in the
form of program transformation: they take the old client program as
input, and produce a new program that preserves the behavior of the
source program as much as possible while targeting the new API.

However, providing a program transformation tool is not easy. Among
the large body of API adaptations performed in practice, only a small
portion has transformation tool support, and it is common for the
transformation tools to introduce bugs in the transformed programs. A
particular type of bugs we are concerned with in this paper is type error,
where a well typed program becomes \emph{not} well typed after the
transformation. For example, Python has provided an official 2to3
script to transform Python programs from Python 2.x to 3.x. However, as
discovered in a case study by Pilgrim and Willison~\cite{python}, the script will introduce a
type error in the transformed code whenever the original code contains a
call to the ``\code{file()}'' method.

% \textbf{ 
%     while python API 2.x was updating to 3.x, a official 
%     update script 2to3 script was provided for developer to automatically
%     update their client programs. A case study
%     \footnote{http://www.diveintopython3.net/case-study-porting-chardet-to-python-3.html}
%     shows that 2to3 script cannot preserve type-safety of updated client programs.
%     Because in python 2.x, there is a method named \textit{file}, and this method
%     was removed in python 3.x. While using 2to3 script to update a client that including
%     method \textit{file}, a type error will be produced in the update client. Our paper
% can solve this problem.}

To overcome the difficulty of providing transformation tools, a large
number of program transformation languages~\cite{twinning, txl, stratego, tom,
survey} have
been proposed. These languages provide high-level constructs for
specifying transformations between programs, reducing the development
cost and preventing certain kinds of errors. For example, a number of
program transformation languages prevent the possibility of
introducing grammatical errors in transformation, either by specifying
the transformation on top of context-free grammars~\cite{txl, stratego}
  or by designing the transformation language specifically for a
programming language~\cite{twinning}. However, as
far as we know, no transformation language for mainstream
object-oriented programs ensures \emph{type safety}: for any
transformation program $p$ and a well typed source program $s$, the
transformed program $p(s)$ is still well typed. As a result, given a
program transformation, we have no guarantee that a well typed
program will still be well typed after the transformation.

It is not easy to ensure type safety in transformation languages. We
highlight two challenges here.  First, typing is one of the most
complex components in modern programming language design, involving
many interleaving of issues. The design of a transformation language
needs to carefully check each intersection of the issues, which is
not an easy job. Second, type safety involves two aspects: correctness
and completeness. Correctness means that every transformed piece in
the program is well typed, while completeness means that all unchanged
pieces are still well typed under the new API. It is easy to ignore one
aspect in transformation language design. As a matter of fact,
Twinning~\cite{twinning}, a modern transformation language for Java
programs, have introduced strict rules for checking types in the
transformation program to prevent the introduction of type 
errors. However, as our motivation section will show later, these
rules still fail to establish full type safety.

% our contribution
%% * a transformation language based on twinning with formal syntax
%% and formal semantics 
%% * a set of type checking rules that guarantee type safetyf
%% * an empirical study that show the feasibility and expressiveness
%% of the language
% paper organization

In this paper we report our first attempt to design a type-safe
transformation language for Java. As the first attempt, we focus on
the class of one-to-many mappings between APIs. One-to-many mappings
mean one method invocation in the source API will be replaced as one
or multiple method invocations in the target API with possible gluing
code. We choose this class for two reasons. 1) One-to-many mappings
are dominant in the migration 
between alternative APIs. An empirical study~\cite{apiupdate}
shows that 95.3\% of the required changes are
one-to-many mapping in the API update of struts, log4j, and jDOM.
% Microsoft summarizes ``the most popular and important'' API mappings
% from Android to iOS as the API mapping
% index\footnote{\url{http://msdn.microsoft.com/en-us/library/windows/apps/jj945420.aspx},
% accessed at Aug 7th, 2014.}.
2) Studying one-to-many mappings is a necessary step toward more
general many-to-many mappings. Since one-to-many mappings are a sub class
of many-to-many mappings, type safety in many-to-many mappings requires type safety
in one-to-many mappings.
As a matter of fact, the language Twinning is
designed for one-to-many mappings, and is known for its simplicity and
usefulness in many adaptation applications. Our approach is built upon
Twinning, where we add extra conditions to ensure type safety. 

More concretely, our contributions are summarized as follows.
\begin{itemize}
\item We propose a new transformation language, SWIN (Safe tWINning),
  for Java program adaptation between alternative APIs. The SWIN
  language is based on Twinning~\cite{twinning}, a modern program
  adaptation language for Java. Compared with Twinning, SWIN includes
  a set of type checking rules to ensure type safety. These
  type checking rules enable a cross-checking over the source API, the
  target API, and the transformation program, and ensure that any well
  typed Java program using the source API will be transformed into a
  well typed Java program using only the target API, if the
  transformation program is well typed under the type checking rules.
  SWIN also have more flexible replacement rules than Twinning.
\item We formalize a core part of SWIN, known as core SWIN. Core SWIN
  works on Featherweight Java (FJ)~\cite{fj}, a formal model of the core Java
  language often used to reason typing-related properties of Java. We
  formally prove the type safety of core SWIN on FJ. We also
  informally describe
  the rest of SWIN and discuss the type safety of full SWIN.
% \item A transformation language, SWIN (Safe tWINning), for Java
%   program adaptation between alternative APIs. The SWIN language is
%   based on Twinning~\cite{twinning}, a modern program adaptation
%   language for Java. The semantics of Twinning is informally
%   specified. SWIN enhanced Twinning with formal semantics and more
%   flexible replacement rules.
% \item A set of type checking rules for SWIN to ensure type
%   safety. These type checking rules enable a cross-checking over the
%   source API, the target API, and the transformation program, and
%   ensure that any well typed Java program using the source API will
%   be transformed into a well typed Java program using only the
%   target API, if the transformation program is well typed under the
%   type checking rules. We formally prove this property on
%   Featherweight Java~\cite{fj}, a formal model of the core Java
%   language often used to reason typing-related properties of Java.
\item We have implemented
  SWIN\footnote{https://github.com/Mestway/SWIN-Project} and have
  evaluated SWIN by implementing three real world transformation
  programs in SWIN. These programs ranges from web APIs~\cite{webapi}
  to local APIs, including both API updating and API switching. Our
  case study shows that SWIN is able to specify a range of useful
  program transformations in practice. More importantly, compared with
  Twinning, the additional type checking rules in SWIN does not confine the
  expressiveness of the language.
\end{itemize}

The rest of our paper is structured as follows. Section~\ref{sec:examples}
briefly introduces Twinning, and then give two motivating examples to show why Twinning 
is not type-safe in program adaptation. 
Section~\ref{sec:examples} also discusses how to maintain type safety 
in program adaptation. 
Section~\ref{tw-fj} presents core SWIN, with an introduction to
Featherweight Java.
% gives an 
% introduction to Featherweight Java, and the formal definition of our 
% transformation language SWIN.
Section~\ref{tw-type} gives the type system for core SWIN, as well as the proof of type safety while transforming programs using rules in SWIN  
on Featherweight Java.
Section~\ref{sec:extension} explains how to extend core SWIN to full SWIN.
% shows that our proof can be extended to full Java.
% In Section~\ref{sec:evaluation}, we select three real cases to show the expressiveness of SWIN, and
% then compare SWIN with Twinning.
Section~\ref{sec:evaluation} presents three case studies which
demonstrate the expressiveness of SWIN.
% Section~\ref{sec:related} shows related work in this area. Conclusions and future work are presented
% in Section~\ref{sec:conclusionsandfutruework}.
Finally, Section~\ref{sec:related} discuss related work and
Section~\ref{sec:conclusionsandfutruework} concludes the paper.

%%% Local Variables: 
%%% mode: latex
%%% TeX-master: "pepm-15"
%%% End: 
