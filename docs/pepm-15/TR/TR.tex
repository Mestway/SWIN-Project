\documentclass[letterpaper]{article}
\usepackage{bussproofs}
\usepackage{amssymb}
\usepackage{fancyhdr}
\usepackage{hyperref}
\usepackage{amsmath}
\usepackage{color}
\usepackage{listings}
\usepackage{verbatim}
\usepackage{bera}
\usepackage{proof,amssymb,enumerate}
\usepackage{amsmath,amsthm}
\usepackage[top=1in, bottom=1in, left=1.25in, right=1.25in]{geometry}

% This is the "centered" symbol
\def\fCenter{{\mbox{\Large$\rightarrow$}}}

\newcommand{\blue}[1] {\textcolor{blue}{#1}}
\newcommand\type[1]{\mathbf{Type}(#1)}
\newcommand{\mychange}{\textcolor{red}{NEW}}
\newcommand{\indentspace}{~~~~~~~~}
\newcommand{\env}[2]{\vdash_{#1}^{#2}}


\newtheorem{theorem}{Theorem}
\newtheorem{lemma}{Lemma}
\newtheorem{definition}{Definition}
\newtheorem{property}{Property}

\usepackage[framemethod=tikz,xcolor=true]{mdframed}

% Optional to turn on the short abbreviations
\EnableBpAbbreviations

% \alwaysRootAtTop  % makes proofs upside down
% \alwaysRootAtBottom % -- this is the default setting

%\pagestyle{fancy}

\begin{document}
\title{Formal Definition of SWIN}
\author{Chenglong Wang \and Jun Li \and Yingfei Xiong \and Zhenjiang Hu}
\date{}
\maketitle

\section{Featherweight Java}
\subsection{Syntax}
\begin{align*}
  & \mbox{Class Declaration}\\
  & \qquad\tt CL\ ::=\  \tt class\ C\ extends\ C \{\bar{C}\ \bar{f}; K\ \bar{M}\}\\
  & \mbox{Constructor Declaration}\\
  & \qquad\tt K \ ::=\  \tt C\ (\bar{C}\ \bar{f})\ \{ super(\bar{f}); this.\bar{f}=\bar{f}\}\\
  & \mbox{Method Declaration}\\
  & \qquad\tt M \ ::=\  \tt C\ m(\bar{C}\ \bar{x})\ \{ return\ t;\}\\
  & \mbox{Term}\\
  & \qquad\tt t \ ::=\  \tt x\ |\ t.f\ |\ t.m(\bar{t})\ |\ new\ C(\bar{t})\ |\ (C)\ t
\end{align*}
\subsection{Type System}
\subsubsection{Subtyping}
\[
\begin{array}{c c}
\infer[\text{(S-SELF)}]
  {\tt C<:C}
  {}
~~~~~~~~~~~~~~~
\infer[\text{(S-TRANS)}]
  {\tt C<:E}
  {\tt C<:D
  &\tt D<:E
  }
\end{array}
\]

\[
\begin{array}{c}

\infer[\text{(S-DEF)}]
  {\tt C<:D}
  {\tt \mathit{CT}(C)=class~C~extends~D~\{...\}
  }

\end{array}
\]

\subsubsection{Auxiliary Functions}
\[
  \begin{array}{c}
    \infer[\text{FIELD-OBJECT}]
    {\tt fields(Object)=\{\}}
    {}
  \end{array}
\]

\[
  \begin{array}{c}
    \infer[\text{(FIELD-LOOKUP)}]
    {\tt fields(C)=\bar{D}\ \bar{g},~ \bar{C}\ \bar{f}}
    {\begin{array}{c}
      \tt \mathit{CT}(C)=class\ C\ extends\ D\ \{\bar{C}\ \bar{f};\ K\ \bar{M}\} \\
      \tt fields(D)=\bar{D}\ \bar{g}
      \end{array}
    }
  \end{array}
\]

\[
  \begin{array}{c}
    \infer[\text{(METHOD-LOOKUP1)}]
    {\tt mtype(m,C)=\bar{B}\rightarrow B}
    {
      \begin{array}{c}
        \tt \mathit{CT}(C)=class\ C\ extends\ D\ \{\bar{C}\ \bar{f};\ K\ \bar{M}\}\\
        \tt B\ m\ (\bar{B}\ \bar{x})\ \{return\ t;\} \in \bar{M}
      \end{array}
    }
  \end{array}
\]

\[
  \begin{array}{c}
    \infer[\text{(METHOD-LOOKUP2)}]
    {\tt mtype(m,C)=mtype(m,D)}
    {
      \begin{array}{c}
        \tt CT(C)=class\ C\ extends\ D\ \{\bar{C}\ \bar{f};\ K\ \bar{M}\}\\
        \text{$\tt m$ is not defined in $\tt\bar{M}$}
      \end{array}
    }
  \end{array}
\]

\[
  \begin{array}{c}
    \infer[\text{(OVERRIDE)}]
    {\tt override(m,D,\bar{C}\rightarrow C_0)}
    {\tt mtype(m,D)=\bar{D}\rightarrow D_0\ implies\ \bar{C}=\bar{D}\ and\ C_0=D_0}
  \end{array}
\]
\subsubsection{Typing}
\begin{center}
\AXC{$\tt x:C\in\Gamma$}
\RightLabel{~(FJ-VAR)}
\UIC{$\tt \Gamma\vdash x:C$}
\DP
\end{center}
\vspace{3pt}

\begin{center}
\AXC{$\tt \Gamma\vdash t_0:C_0$}
\AXC{$\tt fields(C_0)=\bar{C}\ \bar{f}$}
\RightLabel{~(FJ-FIELD)}
\BIC{$\tt \Gamma\vdash t_0.f_i:C_i$}
\DP
\end{center}
\vspace{3pt}

\begin{center}
\AXC{$\tt \Gamma\vdash t_0:C_0$}
\AXC{$\tt mtype(m, C_0)=\bar{D}\rightarrow C$}
\noLine
\BIC{~~~~~~~~~~~~~~~~$\tt \Gamma\vdash \bar{t}:\bar{C}$\qquad$\tt \bar{C}<:\bar{D}$~~~~~~~~~~~~~~~~}
\RightLabel{~(FJ-INVK)}
\UIC{$\tt \Gamma\vdash t_0.m(\bar{t}):C$}
\DP
\end{center}
\vspace{3pt}

\begin{center}
\AXC{$\tt fields(C)=\bar{D}\ \bar{f}$}
\AXC{$\tt \Gamma\vdash \bar{t}:\bar{C}$}
\AXC{$\tt \bar{C}<:\bar{D}$}
\RightLabel{~(FJ-NEW)}
\TIC{$\tt \Gamma\vdash new\ C(\bar{t}):C$}
\DP
\end{center}
\vspace{3pt}

\begin{center}
\AXC{$\tt \Gamma\vdash t_0:D$}
\AXC{$\tt D<:C$}
\RightLabel{~(FJ-UCAST)}
\BIC{$\tt \Gamma\vdash (C)t_0:C$}
\DP
\end{center}
\vspace{3pt}

\begin{center}
\AXC{$\tt \Gamma\vdash t_0:D$}
\AXC{$\tt C<:D$}
\AXC{$\tt C\neq D$}
\RightLabel{~(FJ-DCAST)}
\TIC{$\tt \Gamma\vdash (C)t_0:C$}
\DP
\end{center}
\vspace{3pt}

\begin{center}
\AXC{$\tt \Gamma\vdash t_0:D$ \qquad $\tt C\nless:D$ \qquad $\tt D\nless:C$}
\noLine
\UIC{~~~~~~~~~~~~~~~~${\textrm stupid\ warning}$~~~~~~~~~~~~~~~~}
\RightLabel{~(FJ-SCAST)}
\UIC{$\tt \Gamma\vdash (C)t_0:C$}
\DP
\end{center}
\vspace{3pt}

\begin{center}
\AXC{$\tt \bar{x}:\bar{C}, this:C\vdash t_0:E_0$ \qquad $\tt E_0<:C_0$}
\noLine
\UIC{$\tt CT(C)=class\ C\ extends\ D\ \{...\}$}
\noLine
\UIC{~~~~~~~~~~~$\tt override(m,D,\bar{C}\rightarrow C_0)$~~~~~~~~~~~}
\RightLabel{~(FJ-M-OK)}
\UIC{$\tt C_0\ m\ (\bar{C}\ \bar{x})\ \{return\ t_0;\}\ OK\ in\ C$}
\DP
\end{center}
\vspace{3pt}

\begin{center}
\AXC{$\tt K=C\ (\bar{C}\ \bar{f}) \{super(\bar{f});\ this.\bar{f}=\bar{f}\}$}
\noLine
\UIC{$\tt fields(D)=\bar{D}\ \bar{g}$\qquad $\tt \bar{M}\ OK\ in\ C$}
\RightLabel{~(FJ-C-OK)}
\UIC{$\tt class\ C\ extends\ D\ \{\bar{C}\ \bar{f}; K\ \bar{M}\}\ OK$}
\DP
\end{center}


\section{SWIN}
\subsection{Syntax}
\begin{align*}
  \tt \Pi \quad ::=\quad  &\tt  \{\bar{\pi}\}                     & \text{SWIN program}\\
  \tt \pi \quad ::=\quad  &\tt  (\bar{d})\ [l:C_l\ \rightarrow\ r:C_r]    & \text{rule}\\
  \tt d   \quad ::=\quad  &\tt  x:C_1\hookrightarrow C_2          & \text{variable declaration}\\
  \tt l   \quad ::=\quad  &\tt  x.f ~ | ~ new\ C(\bar{x}) ~|~ x.m(\bar{x})  &\text{code pattern}\\
  \tt r   \quad ::=\quad  &\tt  t                                 & \text{FJ term}
\end{align*}

\subsection{API definition}
\begin{align*}
 \tt API ~ ::=\tt ~ \{ ~ \overline{CL} ~ \}\qquad\text{(API definition)}\\
\end{align*}

\subsection{Evaluation Rules}
\begin{center}
\AXC{$\tt CL=class\ C_{1}\ extends\ C_2\ \{\ \bar{C}_i\ \bar{f}_i;\ K\ \bar{M}\ \}$}                             \RightLabel{~(E-DECLARATION)}
\UIC{$\tt \Pi(CL)= class\ \Pi(C_1)\ extends\ \Pi(C_2)\ \{\ \Pi(\bar{C}_i)\ \bar{f}_i;\ \Pi(K)\  \overline{\Pi(M)}\ \}$}
\DP
\end{center}
\vspace{2pt}

\begin{center}
\AXC{$\tt K=C_1\ (\bar{C}_2\ \bar{f}_2)\ \{super(\bar{f}_3);\ this.\bar{f}_i=\bar{f}_j\}$}                             \RightLabel{~(E-CONSTRUCTOR)}
\UIC{$\tt \Pi(K)=\Pi(C_1)\ (\Pi(\bar{C}_2)\ \bar{f}_2)\ \{super(\bar{f}_3);\ this.\bar{f}_i=\bar{f}_j\}$}
\DP
\end{center}
\vspace{2pt}

\begin{center}
\AXC{$\tt M=C_1\ m(\bar{C}_m\ \bar{x})\ \{return\ t;\}$}                             
\RightLabel{~(E-METHOD)}
\UIC{$\tt\Pi(M)=\Pi(C_1)\ m(\Pi(\bar{C}_m)\ \bar{x})\ \{return\ \Pi(t);\}$}
\DP
\end{center}
\vspace{2pt}

\begin{center}
\AXC{$\tt C_0\hookrightarrow C_1\in \mathbf{TypeMapping}(\Pi)$}                             
\RightLabel{~(E-CLASS)}
\UIC{$\tt \Pi(C_0)=C_1$}
\DP
\end{center}
\vspace{2pt}

\begin{center}
\AXC{$\tt \forall C.~C_0\hookrightarrow C\notin \mathbf{TypeMapping}(\Pi)$}
\RightLabel{~(E-ALTER-CLASS)}
\UIC{$\tt \Pi(C_0)=C_0$}
\DP
\end{center}
\vspace{2pt}

\begin{center}
\AXC{}                 \RightLabel{~(E-T-VALUE)}
\UIC{$\tt \Pi(x)=x$}
\DP
\end{center}
\vspace{2pt}

\begin{center}
\AXC{$\tt (x:C_1\hookrightarrow C_2)[~x.f:C\ \rightarrow\ r:D~]\in \Pi$ ~~~~ $\tt\type{t} <: C_1$}
\noLine
\UIC{$\tt\nexists~(x:C_3\hookrightarrow C_4)[~x.f:C\ \rightarrow\ r:D~]\in \Pi.(\mathbf{Type}(t)<: C_3 <: C_1\land C_3\neq C_1) $}                 \RightLabel{~(E-T-FIELD)}
\UIC{$\tt \Pi(t.f)=\tt[~x\mapsto \Pi(t)~]r$}
\DP
\end{center}
\vspace{2pt}

\begin{center}
\AXC{}      \RightLabel{~(E-T-CAST)}
\UIC{$\tt \Pi((C)\ t)=(\Pi(C))\ \Pi(t)$}
\DP
\end{center}
\vspace{2pt}

\begin{center}
\AXC{$\tt (\bar{d})[\ new\ C_0(\ \bar{x}\ ):C_l\ \rightarrow\ r:C_r]\in \Pi$}
\noLine
\UIC{$\tt\quad\quad \{~\bar{x}:\overline{C_1\hookrightarrow C_2}~\}\subseteq \bar{d} \quad\quad  \type{\bar{t}_u}<:\bar{C}_1$}               
\RightLabel{~(E-T-NEW)}
\UIC{$\tt \Pi(new\ C_0(\bar{t}_u))=[~\bar{x}\mapsto \overline{\Pi(t_u)}~](r)$}
\DP
\end{center}
\vspace{2pt}

\begin{center}
\AXC{$\tt  (\bar{y}:\overline{C_1\hookrightarrow C_2},\ x_0:C_3\hookrightarrow C_4)[\ x_0.m_0(\ \overline{y}\ ):C\ \rightarrow\ r:D]\in\Pi$}
\noLine
\UIC{$\tt \type{t_0}<:C_3\quad\quad \type{\bar{t}_u}<:\bar{C}_1$}
\noLine
\UIC{$\tt \nexists~(\bar{y}:\overline{C_1\hookrightarrow C_2},\ x_0:C_5\hookrightarrow C_6)[\ x_0.m_0(\ \overline{y}\ ):C\ \rightarrow\ r:D]\in\Pi.(\mathbf{Type}(t_0)<: C_5 <: C_3 \land C_5\neq C_3)$}
\RightLabel{~(E-T-INVOKE)}
\UIC{$\tt \Pi(t_0.m_0(\bar{t}_u))=[~x_0\mapsto \Pi(t_0),\ \bar{y}\mapsto \overline{\Pi(t_u)}~](r)$}
\DP
\end{center}
\vspace{2pt}

\begin{center}
\AXC{$\mbox{\tt no other inference rule can be applied}$}   
\RightLabel{~(E-ALTER-NEW)}
\UIC{$\tt \Pi(new\ C_0(\bar{t}_u))=new\ C_0(~\overline{\Pi(t_u)}~)$}
\DP
\end{center}
\vspace{2pt}

\begin{center}
\AXC{$\mbox{\tt no other inference rule can be applied}$}   
\RightLabel{~(E-ALTER-INVOKE)}
\UIC{$\tt \Pi(t_0.m_0(\bar{t}_u))=\Pi(t_0).m(~\overline{\Pi(t_u)}~)$}
\DP
\end{center}
\vspace{2pt}

\begin{center}
\AXC{$\mbox{\tt no other inference rule can be applied}$}                 \RightLabel{~(E-ALTER-FIELD)}
\UIC{$\tt \Pi(t.f)=\tt\Pi(t).f$}
\DP
\end{center}


\subsection{Auxiliary Functions}
\begin{align*}  
&\tt \mathbf{TypeMapping}([(~\bar{x}:\overline{C_1\hookrightarrow C_2}~)\ [l:C_l\ \rightarrow\ r:C_r]]) = \{C_l\hookrightarrow C_r\}\cup\{~\overline{C_1\hookrightarrow C_2}~\}\\
&\tt \mathbf{TypeMapping}(\{\bar{\pi}\}) = \bigcup_{\pi}~(\mathbf{TypeMapping}(\pi)) ~~~~~~ \text{(Extract type migration information)}\\
&\tt \mathbf{Decl}(class~C~extends~D~\{...\}) = C   ~~~~~~ \text{(Extract the declared class name)}
\end{align*}


\subsection{Type Checking Rules}

\begin{center}
  \AXC{$\tt \{~\bar{x}:\bar{C}~\}\vdash_{l}^{API_s} l:C_1$}
  \AXC{$\tt \{~\bar{x}:\bar{D}~\}\vdash_{FJ}^{API_d} r:C_2$}
\RightLabel{~(T-$\pi$)}
\BIC{$\tt (\bar{x}:\overline{C\hookrightarrow D})[l:C_1\rightarrow r:C_2]~\mathit{ok}$}
\DP
\end{center}
\vspace{2pt}

\begin{center}
  \AXC{$\tt \Gamma\vdash_{FJ}^{API} x:C_0$~~~~$\tt mtype(m,C_0)=\bar{D}\rightarrow C$~~~~$\tt \Gamma\vdash_{FJ}^{API}\bar{y}:\bar{D}$}
    \RightLabel{~(T-L1)}
  \UIC{$\tt \Gamma\vdash_{l}^{API} x.m(\bar{y}):C$}
  \DP
\end{center}
\vspace{2pt}

\begin{center}
  \AXC{$\tt fields(C)=\bar{D}~\bar{f}$~~~~$\tt \Gamma\vdash_{FJ}^{API}\bar{x}:\bar{D}$}\RightLabel{~(T-L2)}
  \UIC{$\tt \Gamma\vdash_{l}^{API} new~C(\bar{x}):C$}
  \DP
\end{center}
\vspace{2pt}

\begin{center}
  \AXC{$\tt fields(C)=\bar{D}~\bar{f}$~~~~$\tt \Gamma\vdash_{FJ}^{API} x:C$}\RightLabel{~(T-L3)}
  \UIC{$\tt \Gamma\vdash_{l}^{API} x.f_i:D_i$}
  \DP
\end{center}

\[
\begin{array}{ll}
% Curr
%\tt \mathbf{ClassCover}&\tt(\Pi, API_s, API_d) =\\
%                  &\tt \forall~C_1. (class\ C_1\ extends\ \_~\{ ... \}\in (API_s-API_d) \\
%                  &\tt \indentspace\Rightarrow~ \exists~C_2.(class\ C_2\ extends\ \_\in API_d \land C_1\hookrightarrow C_2\in \mathbf{TypeMapping}(\Pi)))\\
\tt \mathbf{RuleOK} \tt(\Pi) = \forall~\pi. (\pi\in\Pi \Rightarrow \pi~\mathit{ok})\\
\tt \mathbf{ConstrCover} \tt(\Pi, API_s, API_d) =\\
\qquad                  \tt \forall~C_1,\bar{C}. (class\ C_1\ extends\ \_~\{ C_1(\bar{C}\ \bar{\_})~...~\}\in (API_s-API_d) \\
\qquad                  \tt \indentspace\Rightarrow~ \exists~C_2,\bar{C}',\bar{x},r.((~\bar{x}:\overline{C\hookrightarrow C'}~)[new\ C_1(\bar{x}):C_1\rightarrow r:C_2]\in\Pi))\\
\tt \mathbf{MethCover}\tt(\Pi, API_s, API_d) =\\ 
\qquad                  \tt \forall~C_1,C_2,m,\bar{C}.(class\ C_1\ extends\ \_~\{~C_2\ m(~\bar{C}\ \bar{\_}~)\{...\}~...~\}\in (API_s-API_d) \\
\qquad                  \tt \indentspace\Rightarrow~ \exists~x,\bar{y},C'_1,C'_2,\bar{C}',r.((x:C_1\hookrightarrow C'_1,\ ~\bar{y}:\overline{C\hookrightarrow C'}~)[x.m(\bar{y}):C_2\rightarrow r:C'_2] \in \Pi))\\
\tt \mathbf{FieldCover}\tt(\Pi, API_s, API_d) =\\ 
\qquad                  \tt \forall~C_1,C_2,f.(class\ C_1\ extends\ \_~\{C_2\ f;...\}\in (API_s-API_d) \\
\qquad                  \tt \indentspace \Rightarrow~ \exists~x,C'_1,C'_2.((x:C_1\hookrightarrow C'_1~)[x.f:C_2\rightarrow r:C'_2] \in \Pi))                           \\
\tt \mathbf{MapChecking}\tt(\Pi, API_s, API_d) =\\
\qquad                  \tt \forall~C,D.(C\hookrightarrow D\in \mathbf{TypeMapping}(\Pi)\\
\qquad                  \tt \indentspace\Rightarrow~ (\exists~CL\in API_s \cap API_d. (\mathbf{Decl}(CL)=C \land D = C))\\
\qquad                  \tt \indentspace\indentspace \lor (\exists~CL\in API_s-API_d. (\mathbf{Decl}(CL)=C)))\\
\tt \mathbf{Subtyping}\tt(\Pi, API_s, API_d) = \\
\qquad                      \tt\forall~C_i,D_i,C_j,D_j.( C_i\hookrightarrow D_i, C_j\hookrightarrow D_j \in \mathbf{TypeMapping}(\Pi) ~~\Rightarrow~~ (C_i <: C_j\Rightarrow D_i<:D_j))
% \tt \mathbf{TypeSafe} &\tt(\Pi, API_s, API_d) = \\
%                       &\tt \mathbf{RuleOK}(\Pi) \land \mathbf{ConstrCover}(\Pi, API_s, API_d) \land \mathbf{MethCover}(\Pi, API_s, API_d)\\
%                       &\tt \land \mathbf{FieldCover}(\Pi, API_s, API_d) \land \mathbf{MapChecking}(\Pi, API_s, API_d) \land \mathbf{Subtyping}(\Pi, API_s, API_d)
\end{array}
\]

\section{Metatheory}
\begin{lemma}[Typing Context]
Given a SWIN program $\tt\Pi$ acting on $\tt API_s$ to $\tt API_d$, suppose the typing context for a term $\tt t$ is $\tt\Gamma_s = \bar{x}:\bar{C}$ , then the typing context for $\tt\Pi(t)$ is $\tt\Gamma_d=\bar{x}:\overline{\Pi(C)}$. (For simplicity, we use $\tt \Pi(\Gamma_s)$ to represent $\tt \bar{x}:\overline{\Pi(C)}$, i.e. $\tt\Pi$ will act on all variable types in a typing context.)
\end{lemma}
\begin{proof}
According to the FJ typing rules, the typing context will not change once it is created in the rule FJ-M-OK. For the typing context $\tt\Gamma$, except the variable $\tt this$,  all other variables in the typing context are bounded in the definition of a method $\tt M$.

Induction on $\tt\Gamma$. Suppose $\tt\Gamma = \bar{y}:\bar{D}, x:C = \Gamma_1, x:C$, then $\tt\Pi(\Gamma_1)=\bar{y}:\overline{\Pi(D)}$.

There are two cases for $\tt x:C$
\begin{itemize}
\item $\tt x=this$ in $\tt\Gamma$. The type $\tt C$ is a client defined class type, so $\tt C\notin \mathbf{TypeMapping}(\Pi)$. According to the rule E-ALTER-CLASS, $\tt \Pi(C)=C$, then we have $\tt\Pi(\Gamma)=\Pi(\Gamma_0), x:\Pi(C) = \bar{y}:\overline{\Pi(D)}, this:C$.
\item $\tt x$ is an argument in method declaration. According to the rule E-METHOD, after transformation, the type of $\tt x$ in the definition is $\tt\Pi(C)$, thus $\tt\Gamma=\Pi(\Gamma_0), x:\Pi(C) = \bar{y}:\overline{\Pi(D)}, x:\Pi(C)$. 
\end{itemize}

With these two cases proved, the lemma is proved.
\end{proof}

\begin{lemma}[Subtyping]
Suppose $\tt \Pi$ passes SWIN type checking rules, and it transforms an FJ program with $\tt API_s$ to a new program with $\tt API_d$, then:\par
$\tt C_1<: C_2 $ in old program ~~$\Longrightarrow$~~ $\tt\Pi(C_1)<:\Pi(C_2)$ in the transformed program.
\end{lemma}
\begin{proof}
First, we suppose $\tt C_1 <: C_2$, in which $\tt C_1 \neq C_2$ and $\tt\nexists~C$, s.t. $\tt C_1<:C'<:C_2$ and $\tt C'\neq C_1, C'\neq C_2$. 
 
Consider the two possibilities for $\tt C_1$:
\begin{itemize}
\item case-1: class $\tt C_1$ is defined in client code.

In this case, the declaration of $\tt C_1$ should be $\tt CL = class~C_1~extends~C_2~\{...\}$. According to the rule $\tt E-DECLARATION$, we have $\tt \Pi(CL)=class~\Pi(C_1)~extends~\Pi(C_2)~\{...\}$. Thus we have $\tt\Pi(C_1)<:\Pi(C_2)$.

\item case-2: class $\tt C_1$ is defined in API. 

In this case we have $\tt C_2$ is also a API defined class according to the definition of API in FJ. According to the checking rule $\mathbf{ConstrCover}$, these exists $\tt C_1\hookrightarrow D_1, C_2\hookrightarrow D_2 \in \mathbf{TypeMapping}(\Pi)$. By the checking rule $\mathbf{Subtyping}$ and the fact that $\tt C_1<:C_2$, we have $\tt D_1 = \Pi(C_1)<:D_2=\Pi(C_2)$.
\end{itemize}
With this case proved, for any $\tt C_1 <: C_2$, it can be split into $\tt C_1<: C' <: ... <: C_2$. Applying the proof on each step by induction, the lemma is proved.
\end{proof}

\begin{lemma}[Variable Substitution]
Suppose that an FJ term $\tt t$ is well typed under context $\tt \Gamma=\Gamma_1,\{\bar{x}:\bar{C}_x\}$, i.e. $\tt \Gamma\vdash_{FJ} t:C_t$, then after substituting terms $\tt \bar{t}_u$ for variables $\tt \bar{x}$ , with the property that $\tt\Gamma_1\vdash_{FJ}\bar{t}_u:\bar{C}_u$ and $\tt\bar{C}_u <: \bar{C}_x$, $\tt t$ can be typed to $\tt C_t$ or a sub-class of $\tt C_t$. Namely,
$$\tt \Gamma_1,\{\bar{x}:\bar{C}_x\} \vdash_{FJ} t:C_t \Longrightarrow \Gamma_1\vdash_{FJ} [\bar{x}\mapsto\bar{t}_u]t: C'_t,\ C'_t <: C_t$$
\end{lemma}
\begin{proof}
By induction on the derivation on an FJ term $\tt t$, there are five cases to discuss:
\begin{itemize}
\item case-1: $\tt t = x$ and $\tt x$ is a variable.

In this case, we substitute an FJ term $\tt t_u$ for $\tt x$, where $\tt \Gamma_1\vdash_{FJ} t_u:C_u$ and $\tt C_u <: C_t$, then the substitution will be $\tt [x\mapsto t_u]x\Rightarrow t_u$. As $\tt \Gamma_1\vdash_{FJ} t_u:C_u <:C_t$, we have $\tt \Gamma_1\vdash_{FJ}[x\mapsto t_u]x:C_u<:C_t$. Then we have the case proved.

\item case-2: $\tt t=(C)t_1$ and $\tt\Gamma_1,\{\bar{x}:\bar{C}_x\}\vdash_{FJ} t_1:C_1$.

In this case, by induction, we have $\tt \Gamma_1\vdash_{FJ}[\bar{x}\mapsto \bar{t}_u]t_1:C'_1$ and $\tt C'_1 <: C_1$. For the term $\tt t$, after substitution, we have $\tt\Gamma_1\vdash_{FJ}(C)t_1:C$ and $\tt C<:C$. Then we have the case proved.

\item case-3: $\tt t=t_1.f$ and $\tt \Gamma_1,\{\bar{x}:\bar{C}_x\}\vdash_{FJ}t_1:C_1$.

By induction, we have $\tt \Gamma_1\vdash_{FJ} [\bar{x}\mapsto \bar{t}_u]t_1:C'_1$ and $\tt C'_1<:C_1$. Then for field access, it will still access the same field $\tt f$ as it did before. Thus we have $\tt \Gamma_1\vdash_{FJ}[\bar{x}\mapsto \bar{t}_u]t_1.f:C_t$ and $\tt C_t<:C_t$, and the case is proved.

\item case-4: $\tt t=new~C_0(\bar{t}_1)$ and $\tt \Gamma_1,\{\bar{x}:\bar{C}_x\}\vdash_{FJ}\bar{t}_1:\bar{C}_1$.

The substitution $\tt[\bar{x}\mapsto \bar{t}_u]t$ equals to $\tt new~C_0([\bar{x}\mapsto\bar{t}_u]\bar{t}_1)$. 

As the term $\tt t$ is well typed, we have:
\[
    \begin{array}{c c c}
      \infer
      {\tt \Gamma\vdash_{FJ} new~C(\bar{t}_1):C_0}
      {\tt fields(C_0)=\bar{D}~\bar{f}
      &\tt \Gamma\vdash_{FJ} \bar{t}_1:\bar{C}_1
      &\tt \bar{C}_1 <: \bar{D}
      }
    \end{array}
\]

By induction, we have $\tt \Gamma_1\vdash_{FJ} [\bar{x}\mapsto\bar{t}_u]\bar{t}_1:\bar{C}'_1$ and $\tt \bar{C}'_1 <: \bar{C}_1$, then we have the following derivation:
\[
  \begin{array}{c c c}
    \infer
    {\tt \Gamma\vdash_{FJ}new~C([\bar{x}\mapsto\bar{t}_u]\bar{t}_1):C_0}
    {\tt fields(C_0)=\bar{D}~\bar{f}
    &\tt \Gamma\vdash_{FJ}[\bar{x}\mapsto\bar{t}_u]\bar{t}_1:\bar{C}'_1
    &\tt \bar{C}'_1<:\bar{C}_1<:\bar{D}}
  \end{array}
\]

Thus we still have $\tt \Gamma_1\vdash_{FJ}new~C([\bar{x}\mapsto \bar{t}_u]\bar{t}_1):C_0$ and $\tt C_0 <: C_0$. This case is proved.

\item case-5: $\tt t=t_0.m(\bar{t}_1)$ and $\tt \Gamma_1,\{\bar{x}:\bar{C}_x\}\vdash_{FJ}\bar{t}_1:\bar{C}_1,t_0:C_0$.

In this case, the substitution $\tt[\bar{x}\mapsto\bar{t}_u]t$ equals to $\tt ([\bar{x}\mapsto\bar{t}_u]t_0).m([\bar{x}\mapsto\bar{t}_u]\bar{t}_1)$.

As the term $\tt t$ is well typed in FJ type system, we have:
\[
  \begin{array}{c c c c}
    \infer
    {\tt\Gamma\vdash_{FJ}t_0.m(\bar{t}_1):C}
    {\tt \Gamma\vdash_{FJ}t_0:C_0
    &\tt \mathsf{mtype}(m,C_0) = \bar{D}\rightarrow C
    &\tt \Gamma\vdash_{FJ}\bar{t}_1:\bar{C}_1
    &\tt \bar{C}_1 <:\bar{D}
    }
  \end{array}
\]

By induction, we have $\tt\Gamma_1\vdash_{FJ}[\bar{x}\mapsto\bar{t}_u]\bar{t}_1:C'_1, t_0:C'_0$ and $\tt C'_0 <: C_0, \bar{C}'_1 <:\bar{C}_1$.

For the condition that $\tt C'_0 <: C_0$, we have $\tt \mathsf{mtype}(m,C'_0)=\mathsf{mtypem}(m,C_0)$ according to the rule METHOD-LOOKUP2.

With these conditions, we have the following derivation for the new term after substitution:
\[
  \begin{array}{c c c c}
  \infer
  {\tt \Gamma_1\vdash_{FJ}[\bar{x}\mapsto \bar{t}_u]t_0.m([\bar{x}\mapsto \bar{t}_u]\bar{t}_1):C}
  {\tt \Gamma_1\vdash_{FJ}[\bar{x}\mapsto \bar{t}_u]t_0:C'_0
  &\tt \mathsf{mtype}(m, C'_0)=\bar{D}\rightarrow C
  &\tt \Gamma_1\vdash_{FJ}[\bar{x}\mapsto \bar{t}_u]\bar{t}_1:\bar{C}'_t
  &\tt \bar{C}'_1<:\bar{C}_1<:\bar{D}}
  \end{array}
\]

Thus we prove the case that: $\tt Gamma_1\vdash_{FJ}[\bar{x}\mapsto \bar{t}_u]t:C$ and $\tt C<:C$.
\end{itemize}

With these five cases proved, we have the lemma proved.
\end{proof}

\begin{lemma}[Term Formation]
Given a well-typed SWIN program $\tt \Pi$, if a term $\tt t$ in the original typing context can be typed to $\tt C$, then after transformation by $\tt \Pi$, the term is well-typed and its type is a subtype of $\tt \Pi(C)$. i.e.
$$\tt \Gamma_s\vdash^{API_s}_{FJ} t:C \Longrightarrow \Gamma_d\vdash^{API_d}_{FJ} \Pi(t):C',\ where\ C'<:\Pi(C)$$
\end{lemma}
\begin{proof}
By induction on a derivation of a term $\tt t$. At each step of the induction, we assume that the desired property holds for all sub-derivations. We give our proof based on the last step of the derivation, which can only be one of the following five cases:

Before we move on to the cases analysis, we should note that according to Lemma 1, we have the relationship that $\tt\Gamma_d=\Pi(\Gamma_s)$.

\begin{itemize}
\item case-1: $\tt t=x$ and $\tt\Gamma_s\env{FJ}{API_s} x:C$.

In this case, we have that $\tt\Gamma_d\env{FJ}{API_d} x:\Pi(C)$ according to Lemma 1. Then we have $\tt\Gamma_d\env{FJ}{API_d}t:\Pi(C)$ and $\tt \Pi(C)<:\Pi(C)$, and the case is proved.

\item case-2: $\tt t=(C) t_1$ and $\tt \Gamma_s\env{FJ}{API_s}t_1:C_1$.

According to the rule E-T-CAST, $\tt \Pi(t) = \Pi(C)~\Pi(t_1)$.

By induction, we have $\tt\Gamma_d\env{FJ}{API_d}t_1:C'_1$ and $\tt C'_1 <: \Pi(C_1)$.

By the rule FJ-*CAST (represent one of the three cast rules), we have $\tt \Gamma_d\env{FJ}{API_d} (\Pi(C))~\Pi(t_1) : \Pi(C)$, and $\tt\Pi(C)<:\Pi(C)$. Thus the case is proved. 

\item case-3: $\tt t=t_1.f$ and $\tt \Gamma_s\env{FJ}{API_s}t_1:C_1$. In this case, there are further two subcases:
  \begin{itemize}
    \item subcase-1: $\tt C_1$ is declared in $\tt API_d$, i.e. $\tt\exists~CL\in API_s$ and $\tt\mathbf{Decl}(CL)=C_1$. Then we have $\tt \Pi(C_1)$ is declared in $\tt API_d$

      According to the checking rule $\tt \mathbf{FieldCover}$, we have a transformation rule $$\tt \pi=(x:C_1\hookrightarrow C'_1)[x.f:C\hookrightarrow r:C']\in\Pi$$ to transform the field access expressions.

      By the evaluation rule E-T-FIELD, we have $\tt \Pi(t_1.f)=[x\mapsto\Pi(t_1)]r$. According to Lemma 3, we have $\tt\Gamma_d\env{FJ}{API_d}[x\mapsto\Pi(t_1)]r:C''$ and $\tt C'' <: C'$. 

      And by the definition of $\tt \mathbf{TypeMapping}$, there exists $\tt C\hookrightarrow C'$ in $\tt \mathbf{TypeMapping}(\Pi)$. Thus $\tt \Pi(C)=C'$.

      With these properties, we have $\tt \Gamma_d\env{FJ}{API_d} \Gamma_d\env{FJ}{API_d}[x\mapsto t_1]r:C''$ and $\tt C'' <: C' =\Pi(C)$. And the subcase is proved.
    \item subcase-2: $\tt C_1$ is defined in client class declarations.

      In this subcase, the rule will be evaluated by the rule E-ALTER-FIELD. By induction, we have $\tt\Gamma\env{FJ}{API_d}\Pi(t_1):C'_1$ and $\tt C'_1 <:\Pi(C_1)$. Then according to the rule FJ-FIELD and the auxiliary function FIELD-LOOKUP and the evaluation rule E-DECLARATION, we have the following derivation tree:
      \[
          \begin{array}{c c}
          \infer
          {\tt \Gamma\env{FJ}{API_d} \Pi(t_1).f : \Pi(C)}
          {\tt \Gamma\env{FJ}{API_d} \Pi(t_1):C'_1
          & \begin{array}{c c}
              \infer 
              {\tt \Pi(C)~f\in\mathsf{field}(C'_1)}
              {\tt C'_1 <: \Pi(C_1)
              &\tt \Pi(C)~f\in\mathsf{field}(\Pi(C_1))}
            \end{array}
          }
          \end{array}
      \]
      And of course, $\tt \Pi(C)<:\Pi(C)$, thus we have the subcase proved.
  \end{itemize}
  With these two subcases proved, the case for field access is proved.

\item case-4: $\tt t=new~C(\bar{t}_1)$ and $\tt\Gamma\env{FJ}{API_s}\bar{t}_1:\bar{C}_1$.

In this case, we still have two subcases to discuss.
  \begin{itemize}
    \item subcase-1: The class $\tt C$ is declared in $\tt API_s$, i.e. $\tt\exists~CL\in API_s.(\mathbf{Decl}(CL)=C_0$), then $\tt \Pi(C_0)$ should be declared in $\tt API_d$.

    By the checking rule $\tt \mathbf{ConstrCover}$, there exists the following transformation rule to transform this term:
      $$\tt \pi=(\bar{x}:\overline{C_2 \hookrightarrow C'_2})[new~C_0(\bar{x}):C\hookrightarrow r:C']$$

    According to the rule E-T-NEW, we have:
    \[
      \begin{array}{c c}
        \infer
        {\tt \Pi(new~C(\bar{t}_1))=[\bar{x}\mapsto \overline{\Pi(t_1)}]r}
        {\tt (\bar{x}:\overline{C_2 \hookrightarrow C'_2})[new~C_0(\bar{x}):C\hookrightarrow r:C']\in\Pi
        & \begin{array}{c c}
            \infer
            {\tt \mathbf{Type}(\bar{t}_1)<:\bar{C}_2}
            {\tt \mathbf{Type}(\bar{t}_1)=\bar{C}_1
            &\tt \bar{C}_1 <:\bar{C}_2}
          \end{array}}
      \end{array}
    \]

    By Lemma 3, we have $\tt \Gamma_d\env{FJ}{API_d}[\bar{x}\mapsto \overline{\Pi(t)_1}]r : C''$ and $\tt C'' <: C'$. And according to the definition of $\tt \mathbf{TypeMapping}$ and the checking rule $\tt \mathbf{Subtyping}$, we have $\tt \Pi(C)=C'$. 

    Then we have $\tt \Gamma_d\env{FJ}{API_d} \Pi(t):C''$ and $\tt C'' <: \Pi(C)$, and the subcase is proved.

    \item subcase-2: The class $\tt C$ is declared in client code. Then the transformation of the term $\tt t$ should be evaluated according to the rule E-ALTER-NEW, thus we have $\tt \Pi(t)=new~\Pi(C)(\overline{\Pi(t_1)})$.

    To finish the proof of the subcase, we need the following facts:
    \begin{itemize}
      \item According to the rule E-CONSTRUCTOR, the constructor definition is transformed to $\tt \Pi(C)~(\overline{\Pi(C_2)}~\bar{x}) \{...\}$. Thus $\mathsf{fields}(\Pi(C))=\overline{\Pi(C_2)}$
      \item By induction, we have $\tt \Gamma_d\env{FJ}{API_d}\overline{\Pi(t_1)}:\bar{C}'_1$ and $\tt \bar{C}'_1 <: \overline{\Pi(C_1)}$.
      \item By the typing rule FJ-NEW, we have $\tt \bar{C}_1<:\bar{C}_2$. And by Lemma 2, we have $\tt \overline{\Pi(C_1)} <:\overline{\Pi(C_2)}$.
    \end{itemize}
    With these facts, we have the following judgment:
    \[
      \begin{array}{ccc}
      \infer
      {\tt \Gamma_d\env{FJ}{API_d}new~\Pi(C)(\overline{\Pi(t_1)}):\Pi(C)}
      {
       \tt \mathsf{fields}(\Pi(C))=\overline{\Pi(C_2)}
      &\tt \Gamma_d\env{FJ}{API_d}\overline{\Pi(t_1)}:\bar{C}'_1
      &\tt \bar{C}'_1<:\overline{\Pi(C_1)} <:\overline{\Pi(C_2)}
      }
      \end{array}
    \]
    And this proved the subcase.
  \end{itemize} 
  With these two subcases proved, the case for object creation is proved.

  \item case-5: $\tt t = t_1.m(\bar{t}_2)$ and $\tt \Gamma_s\env{FJ}{API_s}t_1:C_1,\bar{t}_2:\bar{C}_2$.

  Similar to the previous one, there are also two subcases for this case.
  \begin{itemize}
    \item subcase-1: the class $\tt C_1$ is declared in $\tt API_s$. i.e. $\tt \exists~CL\in API_s.(\mathbf{Decl}(CL)=C_0)$.

    In this subcase, the term $\tt t_1.m(\bar{t}_2)$ will be transformed by a rule according to E-T-INVOKE as the checking rule $\tt MethCover$ guarantees that there is a rule in $\tt \Pi$ to transform the method (At least a rule exists to transform the method $\tt m$ declared in a parent class of $\tt C$).

    Suppose the transformation rule is the following one (And this one is the closest rule to transform):

    $$\tt (\bar{y}:\overline{C_3\hookrightarrow C'_3}, x:C_4\hookrightarrow C'_4)[x.m(\bar{y}):C\rightarrow r:D]$$

    Then by the rule E-T-INVOKE, the transformation will be :
    \[
      \begin{array}{c}
        \infer
        {\tt \Pi(t_1.m(\bar{t}_2))=[x\mapsto\Pi(t_1), \bar{y}\mapsto\overline{\Pi(t_2)}]r}
        {
          \begin{array}{c c}
            \tt (\bar{y}:\overline{C_3\hookrightarrow C'_3}, x:C_4\hookrightarrow C'_4)[x.m(\bar{y}):C\rightarrow r:D]\in\Pi ~~~~
            \tt \mathbf{Type}(t_1) <: C_4 ~~~~ \mathbf{Type}(\bar{t}_2)<:\bar{C}_3 \\
            \tt \nexists (\bar{y}:\overline{C_3\hookrightarrow C'_3}, x:C_5\hookrightarrow C'_5)[x.m(\bar{y}):C\rightarrow r:D]\in\Pi.(\mathbf{Type}(t_0)<:C_5 \land C_5\neq C_4)
          \end{array}
        }
      \end{array}
    \]

    By Lemma 3, we have $\tt \Gamma_d\env{FJ}{API_d}[x\mapsto\Pi(t_1), \bar{y}\mapsto\overline{\Pi(t_2)}]r:C'$ and $\tt C'<:D$. Also, by the definition of $\tt \mathbf{TypeMapping}$, we have $\tt \Pi(C)=D$. Thus we have $\tt\Gamma\env{FJ}{API_d}t_1.m(\bar{t}_2):C'$ and $\tt C'<:\Pi(C)$. And this subcase is proved.

    \item subcase-2: the class $\tt C_1$ is declared in client code. And in this case, the term $\tt t$ will be transformed by the rule E-ALTER-INVOKE:
    $$\tt \Pi(t_1.m(\bar{t}_2))=\Pi(t_1).m(\overline{\Pi(t_2)})$$

    To finish the proof of this case, we need the following points:
    \begin{itemize}
      \item By induction, we have $\tt \Gamma_d\env{FJ}{API_d}\Pi(t_1):C'_1$ and $\tt C'_1<:\Pi(C_1)$, $\tt \Gamma_d\env{FJ}{API_d}\overline{\Pi(t_2)}:\bar{C}'_2$ and $\tt \bar{C}'_2 <: \overline{\Pi(C_2)}$.
      \item According to the well-typedness of the original term in $\tt API_s$, we have the following derivation:
        \[
          \begin{array}{ccc}
            \infer
            { \tt \Gamma_s\env{FJ}{API_s}t_1.m(\bar{t}_2):C}
            {
               \tt \Gamma_s\env{FJ}{API_s}t_1:C_1
              &\tt \mathsf{mtype}(m,C_1)=\bar{C}_u\rightarrow C
              &\tt \Gamma_s\env{FJ}{API_s}\bar{t}_2:\bar{C}_2
              &\tt \bar{C}_2 <:\bar{C}_u
            }
          \end{array}
        \]
      \item According to the rule E-DECLARATION and E-METHOD, $\tt\mathsf{mtype}(m, \Pi(C_1)=\bar{\Pi(C_u)}\rightarrow \Pi(C)$. By the definition of $\tt\mathsf{mtype}$, we have $\tt \mathsf{mtype}(m,C'_1)=\mathsf{mtype}(m,\Pi(C_1))$.
      \item By the checking rule $\tt \mathbf{Subtyping}$, we have $\tt \overline{\Pi(C_2)}<:\overline{\Pi(C_u)}$
    \end{itemize}

    With these facts, we can derive the type of the transformed term using the following tree:

    \[
      \begin{array}{ccc}
        \infer
        {\tt \Gamma_d\env{FJ}{API_d}\Pi(t_1).m(\overline{\Pi(t_2)}):\Pi(C)}
        {
          \begin{array}{cc}
        \tt \Gamma_d\env{FJ}{API_d}\Pi(t_1):C'_1 ~~~~
        \tt \mathsf{mtype}(m, C'_1)=\mathsf{mtype}(m, \Pi(C_1))=\overline{\Pi(C_u)}\rightarrow\Pi(C)\\
        \tt \Gamma_d\env{FJ}{API_d}\overline{\Pi(t_2)}:\bar{C}'_2~~~~
        \tt \bar{C}'_2<:\overline{\Pi(C_2)}<:\overline{\Pi(C_u)}
          \end{array}
        }
      \end{array}
    \]
    And thus we have $\tt \Gamma_d\env{FJ}{API_d}\Pi(t):\Pi(C)$ in this subcase.
  \end{itemize}
  With these two subcases proved, we have the lemma holds for the case 5.
\end{itemize}
With these five cases for a term proved, by induction, the lemma holds for any FJ term.
\end{proof}

\begin{lemma}[Method Formation]
An FJ method declaration is well formed after transformed by a well-typed SWIN program. i.e. For any \verb|M|, $$\tt \Pi(M)=\Pi(C_1)\ m(\Pi(\bar{C}_m)\ \bar{x})\ \{return\ \Pi(t);\}$$ is well-formed with new API if $\tt \Pi$ is well typed.
\end{lemma}

\begin{proof}
According to Lemma 4, we have $\tt \{\bar{x}:\overline{\Pi({C}_m)}, this:\Pi(C_1)\}\env{FJ}{API_d} t:C'_1$ and $\tt C'_1<:\Pi(C_1)$.

Suppose $\tt CT(C)=class~C~extends~D~\{...\}$, according to the rule E-METHOD and E-DECLRATION, $\tt \mathsf{override}(m,\Pi(D),\overline{\Pi(C)}\rightarrow C_1)$.

Then the formation of the transformed term is proved by the FJ-M-OK derivation on these judgments:
\begin{center}
\AXC{$\tt \{\bar{x}:\overline{\Pi({C}_m)}, this:\Pi(C_1)\}\env{FJ}{API_d} t:C'_1$ ~~~~ $\tt C'_1<:\Pi(C_1)$}
\noLine
\UIC{$\tt CT(\Pi(C))=class\ \Pi(C_1)\ extends\ \Pi(D)\ \{...\}$}
\noLine
\UIC{~~~~~~~~~~~$\tt \mathsf{override}(m,\Pi(D),\overline{\Pi(C)}\rightarrow C_1)$~~~~~~~~~~~}
\RightLabel{~(FJ-M-OK)}
\UIC{$\tt \Pi(C_1)\ m\ (\bar{C}\ \bar{x})\ \{return\ t_0;\}\ OK\ in\ \Pi(C)$}
\DP
\end{center}
\end{proof}


\begin{theorem}[Type-Safety]
Any FJ program is well-typed after a transformation by a well-typed SWIN program $\tt \Pi$. 
i.e. For any \verb|CL|,
 $$\tt \Pi(CL)= class\ \Pi(C_1)\ extends\ \Pi(C_2)\ \{\ \Pi(\bar{C}_i)\ \bar{f}_i;\ \Pi(K)\  \overline{\Pi(M)}\ \}$$
is well-typed with new API if $\tt \Pi$ is well-typed.
\end{theorem}

\begin{proof}
By Lemma 5, we have all method declarations well formed. And by the rule E-CONSTRUCTOR, we have the following derivation:
\[
  \begin{array}{ccc}
    \infer
    {\tt class~\Pi(C)~extends~\Pi(D)~\{\overline{\Pi(C)}~\bar{f};\Pi(K)~\overline{\Pi(M)}\} ~ \mathit{ok}}
    {\tt \Pi(K)=\Pi(C)~(\overline{\Pi(C)}~\bar{f})\{...\}
    &\tt \mathsf{fields}(\Pi(D))=\overline{\Pi(D)}~\bar{g}
    &\tt \overline{\Pi(M)}~\mathit{ok}}
  \end{array}
\]
With this proved, we have the theorem proved, i.e a well-typed SWIN program can transform any FJ program correctly.
\end{proof}

\end{document}
