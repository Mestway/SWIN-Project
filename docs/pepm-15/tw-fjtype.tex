\begin{section}{Feather Weight Java}
\label{sec:FJTyping}

\begin{subsection}{Syntax}
This part presents the syntax for FJ.
\begin{align*}
\tt CL\ ::=&\tt\ class\ C\ extends\ C\ \{\bar{C}\ \bar{f};\ K\ \bar{M}\}\\
\tt K\ ::=&\tt\ C(\bar{C}\ \bar{f})\{super(\bar{f});\ this.\bar{f}=\bar{f};\}\\
\tt M\ ::=&\tt\ C\ m(\bar{C}\ \bar{x}) \{return\ t;\}\\
\tt t\ ::=&\tt x\ |\ t.f\ |\ t.m(\bar{t})\ |\ new\ C(\bar{t})\ |\ (C)\ t\\
\tt v\ ::=&\tt new\ C(\bar{v}) 
\end{align*}
\end{subsection}

\begin{subsection}{Subtyping}
This part presents the derivation of subtype relation in FJ.
\begin{center}
\AXC{}
\RightLabel{~(S-SELF)}
\UIC{$\tt C<:C$}
\DP
\end{center}

\begin{center}
\AXC{$\tt C<:D$}
\AXC{$\tt D<:E$}
\RightLabel{~(S-TRANS)}
\BIC{$\tt C<:E$}
\DP
\end{center}

\begin{center}
\AXC{$\tt CL\ =\ class\ C\ extends\ D\ \{...\}$}
\RightLabel{~(S-DEF)}
\UIC{$\tt C<:D$}
\DP
\end{center}

\end{subsection}

\begin{subsection}{Typing Rules}
The following present the typing rules for FJ term and FJ class
declaration obtained from \cite{tpl}.
\par
Note that CAST rule in FJ type system is divided into three rules. FJ-UCAST and FJ-DCAST are for cast between two classes with subtype relation while FJ-SCAST is the typing rule for cast between two irrelevant classes, which will generate a ``stupid warning'' in the typing progress.

\vspace{3pt}
\begin{center}
\AXC{$\tt x:C\in\Gamma$}
\RightLabel{~(FJ-VAR)}
\UIC{$\tt \Gamma\vdash x:C$}
\DP
\end{center}
\vspace{3pt}

\begin{center}
\AXC{$\tt \Gamma\vdash t_0:C_0$}
\AXC{$\tt fields(C_0)=\bar{C}\ \bar{f}$}
\RightLabel{~(FJ-FIELD)}
\BIC{$\tt \Gamma\vdash t_0.f_i:C_i$}
\DP
\end{center}
\vspace{3pt}

\begin{center}
\AXC{$\tt \Gamma\vdash t_0:C_0$}
\AXC{$\tt mtype(m, C_0)=\bar{D}\rightarrow C$}
\noLine
\BIC{~~~~~~~~~~~~~~~~$\tt \Gamma\vdash \bar{t}:\bar{C}$\qquad$\tt \bar{C}<:\bar{D}$~~~~~~~~~~~~~~~~}
\RightLabel{~(FJ-INVK)}
\UIC{$\tt \Gamma\vdash t_0.m(\bar{t}):C$}
\DP
\end{center}
\vspace{3pt}

\begin{center}
\AXC{$\tt fields(C)=\bar{D}\ \bar{f}$}
\AXC{$\tt \Gamma\vdash \bar{t}:\bar{C}$}
\AXC{$\tt \bar{C}<:\bar{D}$}
\RightLabel{~(FJ-NEW)}
\TIC{$\tt \Gamma\vdash new\ C(\bar{t}):C$}
\DP
\end{center}
\vspace{3pt}

\begin{center}
\AXC{$\tt \Gamma\vdash t_0:D$}
\AXC{$\tt D<:C$}
\RightLabel{~(FJ-UCAST)}
\BIC{$\tt \Gamma\vdash (C)t_0:C$}
\DP
\end{center}
\vspace{3pt}

\begin{center}
\AXC{$\tt \Gamma\vdash t_0:D$}
\AXC{$\tt C<:D$}
\AXC{$\tt C\neq D$}
\RightLabel{~(FJ-DCAST)}
\TIC{$\tt \Gamma\vdash (C)t_0:C$}
\DP
\end{center}
\vspace{3pt}

\begin{center}
\AXC{$\tt \Gamma\vdash t_0:D$ \qquad $\tt C\nless:D$ \qquad $\tt D\nless:C$}
\noLine
\UIC{~~~~~~~~~~~~~~~~${\textrm stupid\ warning}$~~~~~~~~~~~~~~~~}
\RightLabel{~(FJ-SCAST)}
\UIC{$\tt \Gamma\vdash (C)t_0:C$}
\DP
\end{center}
\vspace{3pt}

\begin{center}
\AXC{$\tt \bar{x}:\bar{C}, this:C\vdash t_0:E_0$ \qquad $\tt E_0<:C_0$}
\noLine
\UIC{$\tt CT(C)=class\ C\ extends\ D\ \{...\}$}
\noLine
\UIC{~~~~~~~~~~~$\tt override(m,D,\bar{C}\rightarrow C_0)$~~~~~~~~~~~}
\RightLabel{~(FJ-M-OK)}
\UIC{$\tt C_0\ m\ (\bar{C}\ \bar{x})\ \{return\ t_0;\}\ OK\ in\ C$}
\DP
\end{center}
\vspace{3pt}

\begin{center}
\AXC{$\tt K=C\ (\bar{C}\ \bar{f}) \{super(\bar{f});\ this.\bar{f}=\bar{f}\}$}
\noLine
\UIC{$\tt fields(D)=\bar{D}\ \bar{g}$\qquad $\tt \bar{M}\ OK\ in\ C$}
\RightLabel{~(FJ-C-OK)}
\UIC{$\tt class\ C\ extends\ D\ \{\bar{C}\ \bar{f}; K\ \bar{M}\}\ OK$}
\DP
\end{center}
\end{subsection}

\begin{subsection}{Auxiliary Definition}
This part presents the auxiliary functions used in FJ typing rules.

\begin{center}
\AXC{}\RightLabel{~(FIELD-OBJECT)}
\UIC{$\tt fields(Object)=\{\}$}
\DP
\end{center}
\vspace{3pt}

\begin{center}
\AXC{$\tt CT(C)=class\ C\ extends\ D\ \{\bar{C}\ \bar{f};\ K\ \bar{M}\}$}
\noLine
\UIC{~~~~~~~~~~~~~~~~$\tt fields(D)=\bar{D}\ \bar{g}$~~~~~~~~~~~~~~~~}
\RightLabel{~(FIELD-LOOKUP)}
\UIC{$\tt fields(C)=\bar{D}\ \bar{g},~ \bar{C}\ \bar{f}$}
\DP
\end{center}
\vspace{3pt}

\begin{center}
\AXC{$\tt CT(C)=class\ C\ extends\ D\ \{\bar{C}\ \bar{f};\ K\ \bar{M}\}$}
\noLine
\UIC{$\tt B\ m\ (\bar{B}\ \bar{x})\ \{return\ t;\} \in \bar{M}$}
\RightLabel{~(METHOD-LOOKUP1)}
\UIC{$\tt mtype(m,C)=\bar{B}\rightarrow B$}
\DP
\end{center}
\vspace{3pt}

\begin{center}
\AXC{$\tt CT(C)=class\ C\ extends\ D\ \{\bar{C}\ \bar{f};\ K\ \bar{M}\}$}
\noLine
\UIC{$\tt m$ is not defined in $\tt\bar{M}$}
\RightLabel{~(METHOD-LOOKUP2)}
\UIC{$\tt mtype(m,C)=mtype(m,D)$}
\DP
\end{center}
\vspace{3pt}

\begin{center}
\AXC{$\tt mtype(m,D)=\bar{D}\rightarrow D_0\ implies\ \bar{C}=\bar{D}\ and\ C_0=D_0$}
\RightLabel{~(OVERRIDE)}
\UIC{$\tt override(m,D,\bar{C}\rightarrow C_0)$}
\DP
\end{center}
\vspace{3pt}

\end{subsection}

\end{section}
%%% Local Variables: 
%%% mode: latex
%%% TeX-master: "gpce-2014"
%%% End: 
