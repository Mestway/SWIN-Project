\section{Case Studies}
\label{sec:evaluation}
\subsection{Research Questions}
Since SWIN puts two more conditions on the replacement rules than
Twinning, a natural question to ask is whether these two additional
conditions confine the expressiveness of the language. In other words,
there are programs that can be written in Twinning but not in SWIN,
but are these programs useful in practice? Furthermore, beyond
Twinning, we also want to understand the expressiveness of SWIN in
general. These considerations lead to two research questions.
\begin{enumerate}
\item Does the extra conditions confine the expressiveness of SWIN
  compared with Twinning? %How much expressive is SWIN compared with Twinning?
\item In general, how much expressive is SWIN?
\end{enumerate}

\subsection{Study Setup}
To answer the two research questions, we perform three case
studies. To answer the first research question, we need to compare
SWIN with Twinning. To do this, we repeat a case study in Twinning
that migrate programs from Crimson v1.1.3\footnote{\url{http://xml.apache.org/crimson/}}
to dom4j v1.6.1\footnote{\url{http://www.dom4j.org/}}. Crimson and
dom4j are both Java libraries for manipulating XML files, but Crimson
is no longer supported. Thus, developers may want to migrate programs
from Crimson to dom4j.

To answer the second research question, we further perform two more
case studies, one is about migration from one API to another API, the
other one is to upgrade clients for incompatible API upgrade. More
concretely, we chose the program migration from Twitter4J v4.0.1\footnote{\url{https://github.com/yusuke/twitter4j/}}
to
Sina Weibo Java API v2\footnote{\url{https://code.google.com/p/weibo4j/}}, 
and the client upgrade from Google Calendar
API\footnote{\url{https://developers.google.com/google-apps/calendar/}}
v2 to v3. Twitter4J is a Java wrapper for the RESTful Twitter
API. Sina Weibo is the Chinese counterpart of Twitter, and it provides
an official Java library for accessing its web API. Google Calendar
API is the official Java library for accessing the data in Google
Calendar.

The two case studies of program migration (from Crimson to dom4j, from
Twitter4J to Sina Weibo API) both involve large APIs, and it is
difficult for us to cover the full APIs. In the case study from
Crimson to dom4j, the Twinning authors~\cite{twinning} chose a client
(log4j v1.2.14\footnote{\url{http://logging.apache.org/log4j/1.2/}}) 
and only wrote transformations for the part of the API
covered by the client. We followed the same step as their case
study. In the case study from Twitter4J to Sina Weibo API, we consider
three example clients on manipulating the timeline provided in the example
directory in the Twitter4J source package, and cover only the part of
the API used in these examples.

To perform the case studies, we implemented SWIN in Java using
the Polyglot compiler framework~\cite{glot}. Both our
implementation and all evaluation data are available at the project
web site\footnote{\url{https://github.com/Mestway/SWIN-Project}}.

\subsection{Results}\label{sec:results}


\subsubsection{General Expressiveness}
In total, we wrote 94 rules for the three case studies, each
transforming a method call to the old API into an expression using the
new API. Our rules cover 97\% of the total API methods that needed to
be transformed in the three case studies. This results indicate that,
though our approach deals only with one-to-many mappings, it is able
to perform a significant portion of program adaptation tasks in
practice.

\subsubsection{Comparison with Twinning}
The only uncovered API changes are three method changes in
Google Calendar API, consisting of 3\% of the total API methods that
needs to be transformed. In the three uncovered method changes, one
method splits into several methods, and we need to decide which
new method to replace the original one based on the calling context,
which is not supported in SWIN.

More concretely, method ``\code{EventWho.getAttendeeType()}'' in
Google Calendar v2 returns a string that may contain either
``attendee'' or ``organizer''. Google Calender v3 replaces this method
with two methods: ``\code{boolean getSelf()}'' which returns true when
 ``attendee'' should be returned and ``\code{boolean getOrganizer()}'' which
returns true when ``organizer'' should be returned. To migrate the
client, we may need to transform the code as follows, where
``\code{getSelf()}'' is a client-written method to test whether the
argument is equal to ``attendee'',
\begin{lstlisting}
String attendeeType = attendee.getAttendeeType();
boolean isSelf = isAttendee(attendeeType);  
\end{lstlisting}
into the code as follows.
\begin{lstlisting}
boolean isSelf = attendee.getSelf();  
\end{lstlisting}

This example shows two fundamental limitations of SWIN. First, to
perform the above transformation, we need to match a sequence of
statements and transform them into one method calls. This requires
many-to-one mapping and is not supported by SWIN. Second, we need to
perform a semantic analysis on the implementation code of
\code{isAttendee} to decide whether to transform the code into
\code{getSelf()} or \code{getOrganizer()}. This kind of conditional
transformation is not supported by SWIN.

Clearly, Twinning also has these limitations and cannot handle the
three split methods in Google Calendar API as well. This result
indicates that SWIN is as expressive as Twinning on our three case
studies. Please note that many API classes have sub classes, and thus
the SWIN programs should be much shorter than Twinning, as in Twinning
we need to repeat the rules for the parent class also on each sub
class. % The result also indicates that SWIN is expressive in general,
% being able to handle the transformation of the majority (97\%) of the
% methods in our case studies.

\subsubsection{Interesting Transformation Patterns}
In the implementation of the three case studies, we also found that
many transformations are not direct method replacement, but can still
be expressed in SWIN by flexible use of the transformation rules. We
summarize three patterns below.

\smalltitle{Method $\leftrightarrow$ Constructor} We may need to
map between class constructors and methods, and in SWIN we can
directly specify such a replacement. For example, in the case from
Crimson to dom4j, we write the following piece of code. This program
is in the text form of SWIN, where we use \texttt{->>} to denote $\hookrightarrow$
and \texttt{->} to denote $\rightarrow$.
\begin{lstlisting}
(f : DocumentBuilderFactory ->> DocumentFactory)
[ (f.newDocumentBuilder()):DocumentBuilder ->
    (new SAXReader(f)):SAXReader ]
\end{lstlisting}

\smalltitle{Type Merging} Sometimes a set of classes in the old API
become one class in the new API. In class \texttt{CalendarEvent} in
Google Calendar v2, there is a method \texttt{getTitle()}. Developers
can use this method to acquire the title of a source, but the type of
the title is \texttt{TextConstruct}.  Class \texttt{TextConstruct} is
a wrapper of a string, and there is a method \texttt{getPlainText()}
which returns the internal string. In Google Calendar v3, the class
\texttt{CalendarEvent} becomes \code{Event}, which directly contains a
method \code{getSummary()} to return the string of title. As a result,
we may need to transform a sequence of method invocations
``\code{x.getTitle().getPlainText()}'' into a single invocation ``\code{x.getSummary()}''.

Although such a transformation implies a many-to-many mapping%  seems similar to the many-to-one
% mapping mentioned above
, it can be implemented in SWIN because
\code{TextConstruct} is only used in the return type of
\code{getTitle()} in Google Calendar API. We can consider the API
upgrade as merging classes \code{CalendarEvent} and \code{TextConstruct} into
\code{Event} and merging methods \code{getTitle()} and
\code{getPlainText()} into \code{getSummary()}. As a result, we can
remove the call to \code{getPlainText()} and replace
\code{getPlainText()} with \code{getSummary()}. The rules are as
follows.

\begin{lstlisting}
(x : CalendarEvent ->> Event) 
    [ (x.getTitle()):TextConstruct -> x:Event ]
(l : TextConstruct ->> Event) 
    [ (l.getPlainText()):String 
        -> (l.getSummary()):String ]
\end{lstlisting}

This pattern indicates that though SWIN is design for one-to-many
mappings, many-to-many mappings can also be supported in a limited
form from the flexibility of the rules.

\smalltitle{Type Deletion} A class in the old API may become totally
useless in the new API. In twitter4j, a \code{Twitter} object can be
obtained by first creating a factory \code{TwitterFactory} and then
invoking the \code{getInstance()} method, but in Sina Weibo API class
\code{Weibo}, the counterpart of \code{Twitter}, can be directly
created. In other words, the class \code{TwitterFactory} is
deleted. Similar to the previous case, we may need to merge a sequence
of method invocations ``\code{new TwitterFactory().getInstance()}''
into one single invocation ``\code{new Weibo()}''.

To implement this transformation in SWIN, we use the dummy class
method~\cite{twinning}. We introduce a dummy class \code{NoF} into the
client code to represent the deleted \code{TwitterFactory}. This
dummy class has no class body and can be added to the client code
before the transformation. In this way we can delete a class while
maintaining the type safety. The transformation rules are as follows.
\begin{lstlisting}
()[ (new TwitterFactory()):TwitterFactory 
    -> (new NoF()):NoF ]
(f : TwitterFactory ->> NoF) 
    [ (f.getInstance()):Twitter
     -> (new Weibo()):Weibo ]
\end{lstlisting}
% % In the original publication of Twinning
% % \cite{twinning}, two case studies are performed: (1) adapting programs
% % from Crimson to Dom4j, and (2) adapting programs from Twitter4j to
% % Fb4j. To compare with Twinning, we repeat the first case study on
% % SWIN, and perform 

% % In our paper, we present an type-safe update language for Java:
% % SWIN. In this section, first, we will show powerful expressiveness of
% % using SWIN to update client programs from three real APIs alternative
% % or update.  Second, because we added some constraints on rules to
% % forbid the unsafe rules. The question is whether we expel the useful
% % rules? In this section, we will declare that SWIN is as powerful as
% % twinning in real program adaptation.  Finally, in our experience, we
% % found that there were some typical cases that SWIN cannot solve, and
% % twinning cannot solve either. We will explain these cases in
% % \ref{limofswin}.

% % \subsection{Expressiveness of SWIN}
% % Twinning\cite{twinning} chose two pairs of APIs to do experiment, including 
% % Crimson and Dom4j, as well as Twitter4j and Fb4j. 
% % The former two APIs are Java XML parser, 
% % others are web APIs about social network.
% % In order to explain the ability of SWIN comparing twinning, 
% % we chose Crimson and Dom4j as our experimental subjects.
% % We chose Sina Weibo API as a substitution of Fb4j for the source code of
% % Fb4j is no longer available. 
% % These two pairs of APIs are about API alternative, and 
% % we chose Google calendar API version 2 and version 3 to explain 
% % the ability of SWIN to perform program adaptation in the scope of API update.

% \subsection{From Crimson to dom4j}
% Crimson and dom4j are both large APIs, and it is difficult to write a
% transformation cover both API in full. In the case study off Twinning
% \cite{twinning}, the authors chose a client (log4j v1.2.14) and only
% wrote transformation for the part of API covered by the client. We
% followed the same step as their case study.

% In this case study, we found that each type and method in Crimson has
% exactly one counter part in dom4j, and the transformation can be fully
% expressed in SWIN. A particularly interesting point is that we need to
% map between type constructors and method calls, and in SWIN we can specify
% such a transformation as follows.
% % The same as twinning, we also chose APIs Crimson and Dom4j as our experiment subjects. 
% % We chose log4j logging API (1.2.14) which uses API Crimson, and twinning also chose
% % this client in its experience. 
% % For twinning does not show other clients in the paper, we chose classes that can 
% % parser a XML completely including \textit{IO} classes and nodes processing class.
% % We find all the methods in these classes can be written the update rules easily 
% % from API Crimson to Dom4j with the support of SWIN. 
% % A complex example is in class DocumentBuilder of API Dom4j that
% % a instance of class \texttt{DocumentBuilderFactory} becomes to a parameter of 
% % the constructor of class \texttt{SAXReader}, and this transformation is allowed in SWIN. 
% % The rule is:
% \begin{verbatim}
% (f : DocumentBuilderFactory ->> DocumentFactory)
% [ (f.newDocumentBuilder()):DocumentBuilder ->
%     (new SAXReader(f)):SAXReader ]
% \end{verbatim}

% \subsection{Sina weibo vs. Twitter}
% Twitter4j is an Java wrapper of Twitter official API 
% which can acquire the data of Twitter inc., and users can
% easily develop their own tool related to tweet.
% Weibo is supposed to be Chinese Twitter, and the functionalities of Weibo API is smiliar with
% Twitter API. Sina Weibo also provided a Java wrapper of its web API, and we chose the
% Weibo Java API (in our paper, Weibo API is the same as Weibo Java API) as 
% an alternative of Twitter4j. 
% For these two APIs are huge, we chose all class about "timeline" in the APIs 
% to do our case study. 
% We find that
% most of the updates about the \textit{timeline} methods
% from TWitter4j to Weibo API can be directly written by the support of SWIN.
% All the update rules about \textit{timeline} methods are considered valid in
% according the properties of SWIN.
% Next, we will list some of interesting rules 
% in our experiment to show the expressiveness of SWIN.

% In Twitter4j, the instances of a class \textit{Twitter} is produced by 
% a factory class \textit{TwitterFactory}.
% The corresponding class of \textit{Twitter} Weibo API is class \textit{Weibo},
% there is no such a factory class to produce the instances of class \textit{Weibo}.
% The update rule from class Twitter instance to class Weibo instance will be a 
% method invoke to a constructor, that is:

% \begin{verbatim}
% ()[ (TwitterFactory.getInstance()):Twitter 
%     -> (new Weibo()):Weibo ]
% \end{verbatim}

% Give this example, we want to show the style of terms can be changed without
% influent the type-safety of programs. Next, we will should part of the
% class deletions are allowed in SWIN.
% As we mentioned above, users can call the static method \textit{getInstance} to 
% get a instance of class \textit{Twitter}. In other way, 
% they can also get an instance of
% class \textit{TwitterFactory} first and use it to get an instance of class
% \textit{Twitter}. Because there is no corresponding class in Weibo API which mapping
% to class \textit{TwitterFactory}, we should delete this class first and transform
% the Twitter instance to Weibo instance. In order to delete this class,
% we may introduce an dummy class NOTwFactory mapping to deleting class. 
% Dummy class will be removed after updated, and these do not influent the
% type-safety of programs. The rules are:

% \begin{verbatim}
% ()[ (new TwitterFactory()):TwitterFactory 
%     -> (new NoF()):NoF ]
% (f : TwitterFactory ->> NoF) 
%     [ (f.getInstance()):Twitter
%      -> (new Weibo()):Weibo ]
% \end{verbatim}

% From these rules, we can see there are two statements in Twitter4j client mapping
% to one statement in Weibo client, and
% introducing dummy classes make SWIN can
% solve simple m-to-1 mapping. These transformations are more powerful than refactoring.
% SWIN allows the parameters of a method addition and deletion.
% Both in Twitter4j and Weibo API, there is a method named \textit{getUserTimeline}, but the parameters of these
% methods are not the same. Method \textit{getUserTimeline} in Twitter4j needs two parameters,
% one of type Twitter and another of type long.
% The similar method in Weibo API needs three parameters of type Weibo, String and Paging. 
% In this case, first, we should add an transformation from type long to String, then
% we should add a default value of extra parameter of type Paging to method \textit{getUserTimeline} in Weibo API.
% The update rule for this method:
% \begin{verbatim}
% (x : Twitter ->> Weibo, id : long ->> long)
% [ (x.getUserTimeline(id)):ResponseList ->
%     x.getUserTimeline(String.valueOf(id), 
%         new Paging()):List ]
% \end{verbatim}


% \subsubsection{Google Calendar v2 vs. v3}
% Google calendar open API is provided for users to process the data in google calendar. 
% For some reason, Google update Calendar API version2 to version3. 
% We inspect its
% client library for Java, in total, 59 classes and methods changed in the evolution from 
% v2 to v3. Using SWIN, we can solve 56 changes of them, it means
% more than 95\% classes and methods change can be write as rules in SWIN.  
% The left three changes are about a variable with String type changes to several
% variables with boolean type, and these are hard to be transformed. We find an interesting
% case while doing the experience. 
% In class \textit{CalendarEvent}, there is a method named \textit{getTitle}. Developers can use
% this method to acquire the title of a source, but the type of the title is \textit{TextConstruct}.     
% Class \textit{TextConstruct} is a wrapper of String, and there is a method \textit{getPlainText}
% which can acquire the String of the tile of a source. First, we want to use the \textit{No Class}
% trick to solve this problem, but it will violate the type mapping as a function. Then, we find a 
% more trick approach that make the deleting class \textit{Link} mapping to required class \textit{Event}, 
% and assign the value of class \textit{Event} to variable of class \textit{Link}. The rules are:

% \begin{verbatim}
% (x : CalendarEvent ->> Event) 
%     [ (x.getTitle()):Link -> x:Event ]
% (l : Link ->> Event) 
%     [ (l.getPlainText()):String 
%         -> (l.getSummary()):String ]
% \end{verbatim}

% \subsection{Limitations of SWIN}
% \label{limofswin}
% From experience presented above, we can see
% SWIN is as powerful as twinning.
% All the rules of these three experiments can be find in \todo{cite update calculus}.
% While doing the experience from Crimson to Dom4j, except migrating clients used in
% twinning, we also inspect other clients. In this process, we find a
% typical case that SWIN cannot slove, certainly neither do twinning. This case is mainly
% about context-sensitive.

% In API Crimson, there is a factory class
% \textit{DocumentBuilderFactory} that produce the instance of class
% \textit{DocumentBuilder}. Class \textit{DocumentBuilderFactory} also
% set some field of DocumentBuilder. However, in API Dom4j, the
% corresponding class of \textit{DocumentBuilder} is class
% \textit{SAXReader}, but there is no factory class to produce the
% instance of this class. Developers should use constructor of
% SAXReader, and set the some field like class
% \textit{DocumentBuilderFactory} in constructor. Thereby, while we
% transforming class \textit{DocumentBuilder} to class
% \textit{SAXReader}, we should also get the information of field set up
% by class \textit{DocumentBuilderFactory}. However, SWIN can only match
% one statements at a time according the definition, so we cannot solve
% these context-sensitive cases by SWIN.

% %     \item \textbf{Class splitting}
% %             A class was divided into more than two classes, however, SWIN require that
% %         the type mapping is a function, the same as twinning. Class splitting will
% %         break this property of SWIN.
% % \end{itemize}
% % In our experience, these three cases dominates no more than 10\% of all program updates, so
% % SWIN is still a useful update language in real program update.


%%% Local Variables: 
%%% mode: latex
%%% TeX-master: "pepm-15"
%%% End: 
