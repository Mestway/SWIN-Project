\section{Introduction}
\label{intro}
Preserving the type-safety (or type correctness) while modifying (or updating) Java programs is a hard work, 
especially the modification is not trivial. 
Modifying a program is a kind of software maintenance under some conditions. 
In software engineering, software maintenance 
is the modification of a software product after delivery to correct faults, to improve performance or 
other attributes. Software maintenance is the last stage in software life cycle, and is the most 
important and time-consuming stage. Studies indicate that the cost of maintaining software dominates 
more than 60\% of all cost in software development, the figure can even up to 80\%\cite{verhoef}. 
Software maintenance can be classified into four categories: corrective maintenance, adaptive maintenance, 
perfective maintenance, and preventive maintenance\footnote{http://www.iso.org/iso/catalogue\_detail.htm?csnumber=39064}. 
Adaptive maintenance means 
modification of a software product performed after delivery to keep a software product usable in a 
changed or changing environment. Statistics indicate that adaptive maintenance 
account for 20\% of all maintenances. 

One scenario for adaptive maintenance is that modifying the client programs to adapt to 
new API (Applicable Programming Interface). In software development, nearly all developers 
try to use existing APIs rather than implementing their own APIs. Using existing APIs can improve 
development efficiency and lower the cost of software development. 
However, some APIs may be failed or evolutive. For a failed API, developers need to 
find a new API that has the similar functionalities as a replacement. 
For a evolutive API, 
developers can use the updated API (API after evolution) as a replacement. 
Because the incompatibility may be introduced by new APIs, in both these cases, 
developers need to modify (update) client programs to adapt to the new 
API. These processes of modifying client programs are program adaptation. 
The usual way of program 
adaptation is manually update programs assistant by the search-and-replace tool, 
but it is tedious and error-prone. 

An automatically program adaptation approach is using rule-based language to describe
the modifications, 
then automatically update these programs with the assistant of tools.
A rule-based language is a meta-language includes a set of
rules. Usually a rule includes two parts, one for match, and another for replacement.
While a code block is matching with a rule, this code block will be replaced according
the rule. 
With respect to other automatic program adaptation approaches, 
using rule-based language to update client programs has many merits.
The modifications are flexible, rather than only using the build-in 
modifications. Thus the ability and generality of using rule-based 
language to modify the program is strong. However, one widely criticized 
problem of using rule-based language is high cost of studying the language.
Twinning \cite{twinning} is a very simple rule-based language, and it is easy to learn and write.
In our experience,  the ability of twinning is powerful. 
In our paper, we provide a rule-based language SWIN which based on twinning.
The syntax and semantics of SWIN are similar as twinning, so the merits of easy
to learn and powerful to write have been inherited. 

However, although many merits in twinning, every developers
write their own rules to update programs still a hard work. First,
the developers need to acquire how the differences between new API and old API.
In order to get these information, they should read the API release notes, 
API source code, and etc., all of these
are not easy and time-consuming. Then they should know the client programs very well to write
valid rules. However, a software usually developed by many people, so developers
need to work cooperative, and it is time-consuming. Although developers write rules carefully, 
maybe there are something wrong in rules.
Therefore, after modified programs, 
the developers need to test whether the modified programs are right (type correct or behavior preserving). 
If there
are some type error in the programs, developers need to modify the program again.
All of these processes are not easy and time-consuming.

In order to lower the painful of developers in program adaptation, we provide the rule-based
language SWIN mainly for API vendors.
When the API failed or API updated, the API vendor can
write a set of rules using SWIN to help the client program developers automatically update their 
programs. For the API vendor is familiar with the API, 
so this is an easy job, and the maintenance of rules is
also not hard. 
Many other rule-based languages are focus on updating general object languages. 
These rule-based languages 
usually grammar-oriented
which free of the specific object program language,
thus these rule-based languages are also free of the type of specific object language. 
In our
paper, we consider the type correctness preserving while modifying programs, 
and we focus on a specific programming language -- Java.
There are some works focus
on preserving the type correctness when using the rule-based language to modify
the programs. One [Martin] work chose the lambda calculus as the object language,
but there is a big gap between lambda calculus and high-level programming language.
Although twinning\cite{twinning} chose Java as object language, but they focus on
the type-safety preservation of specific programs modification. In our paper, we want to extend
the rule-based language twinning as Safe tWINning (SWIN) that assure any type-safety Java
programs can be preserved type correctness after using the rules of SWIN to modify the programs.

For our work mainly provide to developer, and developer should write rules which can update any
client programs with their APIs and assure the updated programs are type correct. Our work
provide the some type safe properties of rules. If rules satisfy these properties, 
they can be used to safely
update any client programs which means any type correctness client program is also 
type correctness after being updated.
We also provide a checking algorithm to check if rules satisfy type safe properties.
In our paper, we chose Featherweight Java as our object language to proof that checked rules can 
really preserve the
type correctness of any client programs.

Our contributions are:
\begin{itemize}
\item We provide the formal syntax and semantics of twinning. 
\item We provide a set of type safe properties of rules and a check algorithm to check if update rules are safety.
\item We proof that the correctness of these properties in featherweight Java, and our proof can 
    assure that the checked rules are safety for all the client program.
\item we provide a tool for automatically program update
\end{itemize}

The rest of the paper is structured as follows. In Section \ref{sec:examples}, first, we 
briefly introduce Twinning. Then, we give two motivating examples to show why twinning
cannot preserve type correctness in program adaptation. 
. In the last part of this Section, 
we informally explain our four safe properties that update rules should satisfy with to preserve the 
type correctness.
Section \ref{tw-fj} gives an 
introduction to featherweight Java, and the formal definition of our update language SWIN.
In Section \ref{tw-type}, we give the formal definition 
of safe properties, as well as the proof of type safety while transforming program with
rules satisfying these properties on featherweight Java.
Section \ref{sec:extension} shows that our proof can be extended to full Java.
In Section\ref{evaluation}, we select three real cases to show that expressiveness of SWIN, and
compare SWIN with twinning.
Section \ref{related} shows related work in this area. Conclusions and future work are presented
in Section\ref{conclusionsandfuturework}.



